% 13章
\chapter{付録}
\label{chap:appendix}

%********************************
%図の追加
%\begin{figure}[H]
%	\centering
%	\includegraphics[scale=1]{fig/13/13-2-1.jpg}
%	\caption{図の説明文}
%	\label{fig13-2-1}
%\end{figure}
%
%%表の追加
%\begin{table}[H]
%	\centering
%	\includegraphics[scale=1]{fig/13/t13-2-1.jpg}
%	\caption{表の説明}
%	\label{table13-2-1}
%\end{table}
%********************************

%%%%%%%%%%%%%%%%%%%%%%%%%%%%%%%
\section{システム設計}
%・冗長系としての側面カメラの搭載可否をきちんと議論すべきだった
%・EM機能試験がEnd-to-endからは程遠く、FMで不具合が続出した

%%%%%%%%%%%%%%%%%%%%%%%%%%%%%%%
\section{5.8}
%・動作確認時、受信機の音のみで確認していた
%・パッチアンテナの偽物を掴まされた
%・パラボラアンテナとLNB給電部を繋ぐケーブル長さ不足
%・ロジプロの保守対応無し

%%%%%%%%%%%%%%%%%%%%%%%%%%%%%%%
\section{構体系}
%・組み立て精度が出せない(組み立てが異常に複雑)
%・レールの長さが足りなかった
%・レールの粗さ計測を実施していなかった
%・ハードアノダイズ処理の2度手間が生じた(納期長い)
%・FMでのディプロメントスイッチ不具合
%・FM振動試験時、配線を噛んでしまっていた

%%%%%%%%%%%%%%%%%%%%%%%%%%%%%%%
\section{VHF/UHF展開アンテナ}
%・アンテナが塑性変形する→コンベックス化

%%%%%%%%%%%%%%%%%%%%%%%%%%%%%%%
\section{通信系}
%・インピーダンスマッチングができない(ゲインが低い)

%%%%%%%%%%%%%%%%%%%%%%%%%%%%%%%
\section{C\&DH系}
%・NanoMind電源ピン仕様を誤って発注
%・I2Cの不具合多発
%・COBCへの書込みが不能になり開発が停止
%・衛星のモード切替の実装が後回し
%・HKデータ生成の実装が後回し

%%%%%%%%%%%%%%%%%%%%%%%%%%%%%%%
\section{電源系}
%・CDRにて野田先生の指摘により大幅に仕様変更。CDR後の設計変更はリスクが高かった。
%・ソーラーシミュレータ結果の評価の仕方がわからない
%・松永研ソーラーシミュレータを無断で使用
%・EPS故障時のClyde Space対応の遅さ
%・太陽電池側のインヒビットの要不要がわからない
%・粗サンセンサー機能の開発・実装
%・インヒビットをMOSFETに変更するまでの経緯

%%%%%%%%%%%%%%%%%%%%%%%%%%%%%%%
\section{振動試験}
%・EM振動試験にてアクセス窓方向の不備発見
%・FM振動試験のやり直しの発生(インヒビットスイッチが動作せず)
%・ポッド挿入方向の逆転(振動試験用PODへ入れてみると受け側のレールがなかった)
%・FM再振動試験にて、チャタリング検知回路が故障
%・振動試験用PODのファスナー類を紛失

%%%%%%%%%%%%%%%%%%%%%%%%%%%%%%%
\section{熱真空試験}
%・ベーキングをやらずにEM熱真空試験を実施してしまった

%%%%%%%%%%%%%%%%%%%%%%%%%%%%%%%
\section{連続動作試験}
%・松永研恒温槽の不適切な使用
%・恒温槽試験が意味のある試験だったか?

%%%%%%%%%%%%%%%%%%%%%%%%%%%%%%%
\section{引渡し}
%・リターン側の電圧が安全審査書類の基準値を超過

%%%%%%%%%%%%%%%%%%%%%%%%%%%%%%%
\section{プロジェクトマネジメント}
%・ソフトウェア開発への人員配置が為されていなかった
%・CIB開発の難航
%・スケジュール管理が池谷君が担当するまであまり機能していなかった

%%%%%%%%%%%%%%%%%%%%%%%%%%%%%%%
\section{展開膜}
%・SMAアンテナが統合時不具合
%・薄膜太陽電池の接合部での破断

%%%%%%%%%%%%%%%%%%%%%%%%%%%%%%%
\section{MDC}
%・展開実験などでIMU機能を未試験

%%%%%%%%%%%%%%%%%%%%%%%%%%%%%%%
\section{伸展カメラ部}
%・切り離し機構が後付けで開発された

%%%%%%%%%%%%%%%%%%%%%%%%%%%%%%%
\section{運用}
%・受信報告フォーム/データ仕様の作成・告知の遅れ
%・運用計画書がなく、行き当たりばったりの運用
%・運用メンバーの参加し忘れ(Slackでリマインダ)
