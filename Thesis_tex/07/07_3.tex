\section{不具合解析(岩崎・大本)}

\subsection{衛星動作モード切替に関する不具合}
衛星動作モード切替でsavingモードに入る際の動作はRXPIC,TXPICそれぞれ以下である.\\
\textbf{RXPIC}\\
EPSの電源を落とす.\\
TXPICに西無線のサブ電源を入れるようにUARTで命令\\
西無線に初期設定信号を送信\\
\textbf{TXPIC}\\
CW送信の1パケットが終了し次第,RXPICからのUART命令を実行\\

この動作において,TXPICに西無線のサブ電源を入れる命令の送信と西無線に初期設定信号を送信の間のdelayが適切ではなく,TXPICが西無線のサブ電源を入れる前に西無線に初期設定信号を送信してしまう不具合があった.このため,本衛星では動作モードがsavingに切り替わると西無線の初期設定が適切に行われず,停波してしまうという不具合が発生した.
このミスの原因としては,本衛星に試験用の西無線機が無く,エンジニアリングモデルでのデバッグ作業に西無線が利用できなかったため,モード切替における西無線の初期設定の可否を試験できなかったことにあった.\\
図挿入あとで


\subsection{I2C信号衝突による不具合}
本衛星はRXPIC,TXPIC,OBC(nanomind)が同一のI2Cラインを用いてI2C通信を行う.I2Cの衝突対策は行ったつもりであったが,いい感じにI2Cが衝突すると(現在調査中)RXPIC,TXPIC,OBCのすべてがリセットを繰り返す状態になってしまうことが分かった.本衛星が軌道上で通信が取れなくなったのはこれが原因だと考えられる.
衛星の引渡しまでにこの不具合に気づけなかった原因として,長時間動作試験を行わなかったことが挙げられる.打ち上げ後の地上試験では4日間エンジニアリングモデルを動作させた際にこの現象が確認されたが,衛星引渡しまでの試験では最長でも6時間程度の動作試験しか行わなかった.後継機の開発を行う際は2週間程度の長時間動作試験は行った方がいいらしい.
