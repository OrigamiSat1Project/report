%文責:小出,林,岩崎,井手

\subsection{OBC}
\subsubsection{概要および設計思想(林)}
\begin{itemize}
	\item OBCは地上局から送信したコマンドを指定された時間に本衛星の各コンポーネントに送る(実行する)役割と,HKの生成などを行っている.OBCのプログラムでは,リアルタイムOSを用いており,様々なタスクを切り替えて処理を行うようになっている.
\end{itemize}
\subsubsection{OBCの内部処理}
\textbf{OBC内部処理の全体像(林)}\par
\begin{itemize}
	\item OBCの内部処理では,衛星放出後のテグスの溶断確認などを行う初期運用,HKデータ生成,地上局からアップリンクしたコマンドの実行に関わるコマンド受信を行っている.
	\item OBCの内部処理では,HKデータを地上局から
	\item コマンド受信とHK受信などの詳細は書かずに全体像のみをここではじめに書く
\end{itemize}\par
\textbf{初期運用(小出)}\par
\textbf{HK生成(小出)}\par
\textbf{コマンド受信(林)}\par
コマンド受信では,地上局からアップリンクされたコマンドの内容を指定された時間に各コンポーネントに送信することを行っている.地上局からアップリンクされたコマンドはCIBによってEEPROMの所定の場所(以下,アップリンクコマンド保存アドレス)に保存されている.OBCはアップリンクコマンド保存アドレスから,RAMであるタイムテーブルにコマンド実行時間とコマンドの保存アドレスを記録し,コマンドの実行を管理している.
\begin{enumerate}
	\item アップリンクコマンドのタイムテーブルへの書き込み\par
	OBCは30秒おきにアップリンクコマンド保存アドレスにアクセスし,新しいアップリンクコマンドがないかを確認している.確認方法はアップリンクコマンド保存アドレスに保存されたアップリンクコマンド各々のCRCcheckと呼ばれる部分を確認することである.CRCcheckには,コマンドがどのようなステータスを持っているかをビットで表現しており,表のようになっている.CRCのチェックが行われており,そのコマンドが消去されたコマンドでなく,まだタイムテーブルへ加えられていないコマンドであれば,タイムテーブルに加えるという仕様になっている.タイムテーブルに載せられる情報は,タスクステータ,コマンド保存アドレス,コマンド実行時間である.タスクステータスには,コマンド未格納,コマンド格納済,コマンド未実行,コマンド実行済がある.
	
	\par\item コマンド実行\par
	

\end{enumerate}
\subsubsection{開発中における不具合およびトラブル(小出・林)}
\subsubsection{次回への改善点(小出・林)}

