%文責:小出,林,岩崎,井手

\subsection{OBC}
\subsubsection{概要および設計思想(林)}
\begin{itemize}
	\item OBCは地上局から送信したコマンドを指定された時間に本衛星の各コンポーネントに送る(実行する)役割と,HKの生成などを行っている.OBCのプログラムでは,リアルタイムOSを用いており,様々なタスクを切り替えて処理を行うようになっている.
\end{itemize}
\subsubsection{OBCの内部処理}
\textbf{OBC内部処理の全体像(林)}\par
\begin{itemize}
	\item OBCの内部処理では,衛星放出後のテグスの溶断確認などを行う初期運用,HKデータ生成,地上局からアップリンクしたコマンドの実行に関わるコマンド受信を行っている.
	\item OBCの内部処理では,HKデータを地上局から
	\item コマンド受信とHK受信などの詳細は書かずに全体像のみをここではじめに書く
\end{itemize}\par
\textbf{初期運用(小出)}\par
\textbf{HK生成(小出)}\par
\textbf{コマンド受信(林)}\par
コマンド受信では,地上局からアップリンクされたコマンドの内容を指定された時間に各コンポーネントに送信することを行っている.地上局からアップリンクされたコマンドはCIBによってEEPROMの所定の場所(以下,アップリンクコマンド保存アドレス)に保存されている.OBCはアップリンクコマンド保存アドレスから,RAMであるタイムテーブルにコマンド実行時間とコマンドの保存アドレスを記録し,コマンドの実行を管理している.
\begin{enumerate}
	\item アップリンクコマンドのタイムテーブルへの書き込み\par
	OBCは30秒おきにアップリンクコマンド保存アドレスにアクセスし,新しいアップリンクコマンドがないかを確認している.確認方法はアップリンクコマンド保存アドレスに保存されたアップリンクコマンド各々のCRCcheckと呼ばれる部分を確認することである.CRCcheckには,コマンドがどのようなステータスを持っているかをビットで表現しており,表\ref{crccheck_contents}のようになっている.7bit目のCRCのチェックがok,3bit目のタイムテーブル格納がno,0bit目の消去済みコマンドがnoであればタイムテーブルに加えるという仕様になっている.タイムテーブルに載せられる情報は,タスクステータ,コマンド保存アドレス,コマンド実行時間である.タスクステータスには,コマンド未格納,コマンド格納済,コマンド未実行,コマンド実行済がある.タスクステータスがコマンド未格納かコマンド実行済であるタスクテーブルにコマンドを追加する仕様になっている.また,コマンドが加えられたタスクテーブルのタスクステータスにはコマンド格納済のステータスが入れられる.
	\begin{table}[hbtp]
		\caption{CRCcheckの内容}
		\label{crccheck_contents}
		\centering
		\begin{tabular}{ccc}
			\hline
			bit番号  & 内容  &  判定  \\
			\hline \hline
			7  & GS--RXCOBC uplink CRCcheck  & 0: error 1:ok \\
			6  & RXCOBC--TXCOBC UART  CRCcheck   & 0: error 1:ok \\
			5  & mainEEPROM--TXCOBC  I2C CRCcheck  & 0: error 1:ok \\
			4  & subEEPROM--TXCOBC  I2C CRCcheck  & 0: error 1:ok \\
			3  & task added to commandtable  & 0:no  1:yes  \\
			2  & task started   & 0:no  1:yes \\
			1  & task finished  & 0:no  1:yes \\
			0  & task discarded  & 0:no  1:yes \\
			\hline
		\end{tabular}
	\end{table}
	\par\item コマンド実行\par
	タスクテーブルに加えられたコマンドが指定された時間にコマンドが実行されるよう,コマンド実行では1秒おきにタスクテーブル内すべてのコマンドの実行時間(タスクタイミング)を確認している.まず,実行時間を確認する前に,タスクテーブル内コマンドのタスクステータスの確認を行う.タスクステータスが,コマンド未格納,コマンド実行済でなくて,コマンド格納済であれば,そのコマンドのタスクタイミングと現時刻を比較し,現時刻がタスクタイミングより進んでおり,1分いないであればコマンド実行に移る.\par
	タスクタイミングの条件を満たしたものはまず最初にCRCの確認を行う.そのコマンドのアップリンクコマンドアドレスにアクセスし,CRCの確認をして問題がなければ,二重コマンドの確認を行う.(二重コマンドの説明は後で述べる.)CRCに問題があれば,OBCの再起動を行いアップリンクコマンド保存アドレスのCRCcheckの0bit目の消去済みコマンドをyesにし,タスクテーブル全部のタスクステータスをコマンド未格納にする.二重コマンドでなければ,各コンポーネントに送信するときの構造体にコマンドの内容を入れ,そのコマンドのタスクステータスをコマンド実行済にする.そのあとは,アップリンクコマンド内のタスクターゲットを確認し,指定されたコンポーネントにコマンドのパラメータを送信している.また,コマンドを実行する際は,HK生成タスクを一時停止させ,	コマンド実行では,タスクテーブルに加えられたコマンドが実行時間になったら,コマンドを実行するコンポーネントに送信する.\par
	送信した後は,コマンドを実行した結果が返り値として帰ってくるようになっている.正常にコマンドが実行されれば返り値が0,ミッションに深刻な影響を与えない異常が出た場合は返り値が正の値,ミッションに深刻な影響を与える異常が出た場合は返り値が負の値になっている.返り値が負の値であった場合,OBCの再起動を行いアップリンクコマンド保存アドレスのCRCcheckの0bit目の消去済みコマンドをyesにし,タスクテーブル全部のタスクステータスをコマンド未格納にする.また,実行したコマンドのコマンド識別IDとコマンドを実行したとき返り値はEEPROMの所定のアドレスに保存され,履歴として残される.最後に実行したコマンドのコマンド識別IDとコマンド返り値の情報があるアドレスもまた,EEPROMの所定のアドレスに保存される.
	
	\par ※二重コマンドについて\par
	二重コマンドとは,実行されてしまうと衛星に重大な問題を引き起こすコマンドを,ヒューマンエラーにより誤ってアップリンクされても実行されないための仕組みである.\par
	二重コマンドを実行するためには,2回アップリンクを行う必要がある.最初にアップリンクするのはEEPROMの所定のアドレスにある二重コマンド格納アドレスである.次にアップリンクするのは,アップリンク保存アドレスである.2回目にアップリンクしたコマンドは通常のコマンドと同様にタスクテーブルに入れられ,コマンド実行時間になったら,実行する手順に入る.また,二つのコマンドは,コマンド内容は全く同じにする必要がある.上でも述べたように,二重コマンドか否かを判断するのはCRCの確認を行った後である.コマンドのタスクターゲットがOBCの時に,二重コマンド確認に入る.この時,二重コマンド格納アドレスのアップリンクコマンドのCRCを確認して問題がなければ,実行手順に入っているコマンドと二重コマンド格納アドレスのコマンドのコマンド内容を比較し,まったく同じであれば,アップリンクコマンドのコマンド内容の各byteをそれぞれ2byteずらして新たにコマンドを生成する.その手順を経て,タスクターゲットが地上局から生成できない二重コマンド用のものに切り替わり,二重コマンドを実行することができるようになっている.	

\end{enumerate}
\subsubsection{開発中における不具合およびトラブル(小出・林)}
\begin{itemize}
	\item SDカードのマウントの変更がOBCの起動中に行えると考えていたが,起動時にしかマウントの変更ができないことが発覚.GOMSPACEにも問い合わせたが,物理的には可能だが,推奨していないと言われた.
\end{itemize}
\subsubsection{次回への改善点(小出・林)}
\begin{itemize}
	\item 全体的にプログラムの開発が遅かった.バス部なので本番でもきちんと動くものを早い段階で完成させておくべきだった.コマンド実行のプロセスも早いうちに完成させておくべきだったと思う.
\end{itemize}
