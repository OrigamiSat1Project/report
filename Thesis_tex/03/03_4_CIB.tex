%文責:中塚・黒崎

\subsection{CIB}
\subsubsection{基本設計思想(中塚)}
RXPIC,TXPICは神だよ.外部からはリセットを基本しないよ.週1回だけのリセットだけだよ.その他あれば.

\subsubsection{プログラム概要(黒崎)}
\textbf{CIB全体}
\begin{itemize}
	\item 初期運用やってるよ.詳細は後述.初期運用専用の節を作成してしまってもいいのでは.
	\item 初期運用モードと通常運用モード違い
\end{itemize}

\textbf{RXPIC役割}
\begin{itemize}
	\item モード切替判断.詳細は後述.モード切替専用の節を作成してしまってもいいのでは.
	\item アップリンク待機
\end{itemize}

\textbf{TXPIC役割}
\begin{itemize}
	\item CW HKデータ垂れ流し(HKデータダウンリンクの
	中身専用(CIB/OBC共通)の節を作成してしまってもいいのでは.)
	\item 指示があればFM HKデータダウンリンク(HKデータダウンリンクの
	中身専用(CIB/OBC共通)の節を作成してしまってもいいのでは.)
	\item CWFMデータダウンリンク
\end{itemize}

\subsubsection{RXPIC詳細(3月中に間に合えば黒崎.無理なら中塚)}
ソフトの中身やswitch文の中をそれぞれ詳しく

\subsubsection{TXPIC詳細(中塚)}
ソフトの中身やswitch文の中をそれぞれ詳しく

\subsubsection{デバックや機能確認に関して(中塚・黒崎)}
\begin{itemize}
	\item 振動試験等の前後には機能確認しよう
	\item EMFMそれぞれでちゃんんと機能確認しよう
	\item FMは組み立てする度に機能確認しよう
	\item どんな機能確認をしていたか
	\item その他注意点等
\end{itemize}

\subsubsection{コメントや次回への改善点(中塚・黒崎)}
\hspace{2ex}
\textbf{CIB全体}
\begin{itemize}
	\item aaaa
\end{itemize}

\hspace{2ex}
\textbf{RXPIC}
\begin{itemize}
	\item aaaa
\end{itemize}

\hspace{2ex}
\textbf{TXPIC}
\begin{itemize}
	\item aaaa
\end{itemize}
