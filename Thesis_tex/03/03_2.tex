\section{通信系 (衛星) (大本)}

本衛星に搭載する通信機としては,以下の3つがある.
\begin{itemize}
	\item {1} 地上局からのアップリンクを受信するVHF系
	\item {2} 地上局へCW信号,FM信号をダウンリンクするUHF系
	\item {3} 地上局へ大容量データをダウンリンクする5.8GHz系
\end{itemize}

衛星と地上局の通信の概略図を図\ref{fig4-2-1}に示す.
本衛星の回線設計は本衛星の軌道情報,東工大松永研究室地上局設備の性能を加味し,図\ref{fig4-2-2}のようになされた.
それぞれに要求される性能を記述する.
\begin{figure}[H]
	\centering
	\includegraphics[scale=0.5]{03/fig/4-2-1.jpg}
	\caption{通信系概略図}
	\label{fig4-2-1}
\end{figure}
\begin{figure}[H]
	\centering
	\includegraphics[scale=0.4]{03/fig/4-2-2.jpg}
	\caption{回線設計}
	\label{fig4-2-2}
\end{figure}

\subsection{VHF系}
VHF帯(Very High Frequency)の通信機では地上局からのアップリンクを受信するために用いる.通信機は西無線研究所の301A型を用いた.
アンテナにはコンベックス加工を行ったリン青銅製のモノポールアンテナ(幅5mm, 厚さ0.1mm)のものを用い,長さはインピーダンスマッチング試験を通じて決定した.
以下に設計スペックおよび系統図(\ref{fig4-2-3})を示す.
\begin{itemize}
	\item 周波数:145.980MHz
	\item 送信出力:100mW(CW),800mW(CW)
	\item 寸法:60x50x10.5mm
	\item 重量:38g
	\item アンテナ利得:0dBi
	\item 周波数帯域幅:500Hz(CW),20kHz(FM)
\end{itemize}
\begin{figure}[H]
	\centering
	\includegraphics[scale=0.6]{03/fig/4-2-3.jpg}
	\caption{UHF/VHF系通信系統図}
	\label{fig4-2-3}
\end{figure}

\subsection{UHF系}
UHF帯(Ultra High Frequency)の通信機では地上局へCW信号,FM信号をダウンリンクするために用いる.通信機は西無線研究所の301A型を用いた.
アンテナにはコンベックス加工を行ったリン青銅のモノポールアンテナ(幅5mm, 厚さ0.1mm)のものを用い,長さはインピーダンスマッチング試験を通じて決定した.
以下に設計スペックおよび系統図(\ref{fig4-2-3})を示す.
\begin{itemize}
	\item 周波数:437.505MHz
	\item 寸法:100x60x10.5mm
	\item 重量:60g
	\item アンテナ利得:0dBi
\end{itemize}

\subsection{5.8GHz系}
5.8GHz系の通信機では画像などの大容量データをダウンリンクするために用いる.通信機はロジカルプロダクト社製 LPTX5840-1を用いた.
アンテナには円偏波パッチアンテナ(30x30x1.6mm)を用いた.
以下に設計スペックおよび系統図(\ref{fig4-2-4})を示す.
\begin{itemize}
	\item 周波数:5840MHz
	\item 送信出力:2W
	\item 寸法:76x70x16mm
	\item 重量:220g
	\item アンテナ利得:3dBi
	\item 周波数帯域幅:210kHz
\end{itemize}
\begin{figure}[H]
	\centering
	\includegraphics[scale=0.6]{03/fig/4-2-4.jpg}
	\caption{5.84GHz系通信系統図}
	\label{fig4-2-4}
\end{figure}
