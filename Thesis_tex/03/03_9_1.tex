\section{ミッション系}
\subsection{5.8GHz通信ミッション(井手)}
ここでは5.84GHz帯高速通信を担うモジュール(以下5.8)について述べる.
\subsubsection{ミッション内容}
衛星内部で膜展開前後の写真・動画を地上局に送る.モジュールは福岡にあるロジカルプロダクト社から購入しており,これは福岡工業大学の衛星プロジェクトで用いたものと同じである.

\subsubsection{ハードの概要}
5.8にはMicrochip Technology社のPIC16F886(以下PIC)を用いており,写真保存用のフラッシュメモリ(以下FROM)はMicron Technology社のM25P32を用いている.ここに回路図の写真.

\subsubsection{ソフトの概要}
PICのプログラムにはC言語を用いている.衛星にはブランチ名master-debugger,コミット番号51237d7e783d32032fe3db0332d3c69a6ab9e13eを書き込んだ.ソフトは1つのメイン文が繰り返し実行されるが,主に2つの部分に分かれている.1つはコマンドをOBCから受け取る部分,もう1つは受け取ったコマンドを実行する部分である.

まずコマンドの受け取り方法について記述する.コマンドはいつOBCから送られてくるかわからないため常に待機状態にあり,CRC16を用いたコマンドのチェックを通過したコマンドのみ実行する.OBCとの通信にはUARTを用いた.通信速度はモジュールの振動子から14400, 57600, 115200[bps]と設定できるが,デフォルトは115200に設定してある.OBCとの通信時には常に115200を用い,地上局との通信時には通信速度を変更することができる.