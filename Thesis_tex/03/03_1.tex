\section{電源系 (概要/EPS/インヒビット設計(二重絶縁)/電源系統図/電池/SAP)(池谷・中塚)}
\subsection{概要}
本節では本衛星の電源系について述べる.本衛星の電源系の概要を図\ref{3_1_power_diagram}に示す.本衛星の電源系は主に以下のコンポーネントから構成されている.
\begin{itemize}
	\item 太陽電池パネル(Solar Array Panel, SAP)
	\item 電源基板EPS
	\item バッテリ
	\item CIB電源系
	\item 伸展カメラ部電源系
\end{itemize}


% begin{landscape}
% begin{figure}[htbp]
% 	\begin{center}
% 		\includegraphics[width=0.5\linewidth]{./03/fig/Power_diagram.png}
%		\caption{Example of a figure caption.}
%		\label{3_1_power_diagram}
%	\end{center}
% \end{figure}
% \end{landscape}            


\subsection{SAP}

\subsubsection{SAP試験}

\subsection{バッテリ}
バッテリはClyde Space社の30Whr Standalone CubeSat Batteryを購入した(図\ref{fig3_1_bat}).バッテリの電気・構造的特性を表\ref{table3_1_bat_spec}に,絶対最大定格を表\ref{table3_1_bat_max}に示す.
公称 7.6V, 放電容量3900mAhr,
2s3p リチウムポリマー電池
UN勧告適合品,NASA標準EP-Wi-
032適合品.

過放電,過充電,内部短絡,外部短
絡保護回路が組み込まれたバッテリ
パック(次ページ).

2つの保護機能については,メーカーから
試験報告書を入手.
上記とは別途,OrigamiSat-1開発チームで
環境試験(振動,衝撃)前後で特性測定の
検査を実施する.

\begin{figure}[htbp]
	\begin{center}
		\includegraphics[width=0.5\linewidth]{./03/fig/battery.jpg}
		\caption{30Whr Standalone CubeSat Battery (c)Clyde Space}
		\label{fig3_1_bat}
	\end{center}
\end{figure}

\begin{table}[htbp]
	\begin{center}
		\includegraphics[width=0.8\linewidth]{./03/fig/battery_spec.png}
		\caption{30Whr}
		\label{table3_1_bat_spec}
	\end{center}
\end{table}


\begin{table}
	\caption{Maximum Ratings of the Battery}
	\label{table3_1_bat_max}
	\centering
	\begin{tabular}{ccccc}
		\hline \hline
		\multicolumn{5}{c}{Max Ratings Over Operating Temperature Range (Unless Otherwise Stated)}\\
		&&BCR  & Value  &  Unit  \\
		\hline
		\multirow{3}{*}{Charge Limits}&Voltage&max&8.4&V\\
		&Current&max&6&A\\
		&Current Rate& max &1.53C &Fraction of Capacity\\
		\hline
		\multirow{3}{*}{Discharge Limits}&Voltage&max&6.2&V\\
		&Current&max&6&A\\
		&Current Rate& max &1.53C &Fraction of Capacity\\
		\hline
		\multicolumn{2}{c}{Operating Temperature}&\multicolumn{2}{c}{ -10 to 50} & °C\\
		\multicolumn{2}{c}{\multirow{3}{*}{Storage Temperature}}&\multicolumn{2}{c}{1 Year: -20 to +20}&\multirow{3}{*}{°C}\\
		&&\multicolumn{2}{c}{3 Months: -20 to +45}&\\
		&&\multicolumn{2}{c}{1 Month: -20 to +60}&\\
		\multicolumn{2}{c}{Vacuum}&\multicolumn{2}{c}{10-5}&torr\\
		\multicolumn{2}{c}{Vibration}&\multicolumn{2}{c}{To [RD-3]}\\
		\hline
	\end{tabular}
\end{table}
	

本バッテリは内部保護機能を有しており,


(1) Cell Level Protection Circuit
• 試験結果はCS社提供(「Test Report for Lot
Acceptance Testing」).
• セルごとに配置され,セルは2直列になって
いるので,ユニットとしては2つの保護機能
を有する.

(2) Over-current Polyswitch Protection
(resettable fuse)
• 試験結果(OP-S1-0021 電池セル仕様書、
及び購入先認証状)を入手


\begin{figure}[htbp]
	\begin{center}
		\includegraphics[width=0.5\linewidth]{./03/fig/bat_protection.png}
		\caption{Integrated EPS and Battery Protection Architecture 転載}
		\label{mir}
	\end{center}
\end{figure}

\begin{figure}[htbp]
	\begin{center}
		\includegraphics[width=0.5\linewidth]{./03/fig/cell_protection.png}
		\caption{Integrated EPS and Battery Protection Architecture 転載}
		\label{cell_p}
	\end{center}
\end{figure}


\subsection{CIB電源系}

\begin{figure}[htbp]
	\begin{minipage}{0.5\hsize}
		\begin{center}
			\includegraphics[width=0.7
			\linewidth]{./03/fig/CIB_1.jpg}
		\end{center}
	\end{minipage}
	\begin{minipage}{0.5\hsize}
		\begin{center}
			\includegraphics[width=0.7\linewidth]{./03/fig/CIB_2.jpg}
		\end{center}
	\end{minipage}\\		
	\begin{center}
		\caption{CIB}
	\end{center}
\label{CIB}
\end{figure}

\subsubsection{インヒビット回路}
イプシロンロケット


要求により

以下のハザードを設けられた
これらのハザードに対応するために
3インヒビット回路を設けた.

購入品ではこれらの要求を満たせなかったため新たに

Battery-EPS間に新たに

インヒビット回路の回路図は

のようになっている

SAP-EPS間の遮断


\subsubsection{フライトピン}

\subsubsection{CIB内電源回路}
RXCOBCおよびTXCOBCは
UHF/VHF無線機
はほとんどの場合起動していないければならない

そこで新たな電源回路を設けた

三端子レギュレータを並列で繋いだ

そこで二つのレギュレータの出力電流の違いによる
逆流を防止するために
ORダイオード
ダイオードによる電圧降下を防ぐために


また12V系は
突入電流が
購入EPSの

そこで新たにDC-DCコンバータ

これらは二重に

さらに過電流対策

放射線試験によるICの放射線耐性を




\subsection{EPS}
EPSは

を購入した
\begin{figure}[htbp]
	\begin{center}
		\includegraphics[width=0.5\linewidth]{./03/fig/eps.jpg}
		\caption{EPS}
		\label{eps}
	\end{center}
\end{figure}

\subsection{ミッション部電源系}
ミッション部電源系は
Raspberry Pi
が受け取ったUART信号により
スイッチのON/OFFを切り替える
\ref{}にて
後述する


