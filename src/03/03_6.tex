\section{構体系(奥山・大野) 重量管理も含む}

\if0
奥山
・工場の人と仲良くして納期短縮
・放出検知ピン(二硫化モリブデン,PEEK)
・基板保持ねじれ
・EMの不具合
・基板保持の治具
・精度出しの方法どうやって決めたか
・組立手順どう決めたか
・発注
・サーミスタ
・配線

ひらくさん
・組み立て精度が出せない(組み立てが異常に複雑)
・レールの長さが足りなかった →形状計測
・レールの粗さ計測を実施していなかった →形状計測
・ハードアノダイズ処理の2度手間が生じた(納期長い)
・FMでのディプロメントスイッチ不具合
・FM振動試験時、配線を噛んでしまっていた
  
  ----------------------------------------
  CAD作成
  ・干渉チェック
  ピッタリは危険(展開アンテナはまったやつ)
  ・EMの不具合:FM発注リスト参照
  ・CAD非反映物(ハーネス、コネクタ等)の考慮
  ・ハーネスの通り道にR
  ・ディプロイメントスイッチななめる
  ・工具が入るか パッチ土台とか
  
  図面作成
  ・加工時見やすいように
  ・公差の指定
  ・加工順序考える
  ・早めに工場と相談
  
  発注
  ・業者と工場の違い
  ・納期注意
  ・誰が見ても誤解しない図面
  ・アルマイト時間かかる マスキングとか
  
  
  加工
  ・ハードアノダイズ処理の2度手間が生じた(納期長い)
  ・基板保持ねじれ 組み立て?
  
  組み立て
  ・基板保持の治具
  ・基板入ったりはいらなかったり、、
  ・サーミスタ
  ・FM振動試験時、配線を噛んでしまっていた
  ・配線
  ・精度出しの方法どうやって決めたか
  ・組立手順どう決めたか
  ・組み立て精度が出せない(組み立てが異常に複雑)
  ・FMでのディプロメントスイッチ不具合
  ・手順書 誰でも組み立てられるように
  とくにバスはソフトの人もできるようにしておくべき
  全工程ソフトの人と確認
  ・工具は毎回しまう
  
\fi

本稿では,私が携わったEMの設計の修正からFMの組立までについて,各開発フェーズごとに記載する.
なお,詳細な不具合は「OP-S1-0065\_OrigamiSat-1不具合・A/I管理表\_20190124」に記載している.また,設計についてEMからFMで修正する際の修正項目(EMでの不具合項目)は「FM発注リスト.xlsx」に記載している.


\subsection{3DCAD設計}
3DCAD設計でミスが起こると開発が進んだときの手戻りが大きいので気をつける.
\subsubsection{干渉の確認}
構体の部品同士の干渉を確認する必要がある.ここで注意したいことが,加工公差である.CAD上では公差は指定していないため,公差を含めて干渉がないかを考慮する必要がある.EMでは,展開アンテナとそれを通す溝の幅が同じであったため,CAD上では干渉はないが,実際にはアンテナがはまって展開できなかった.(追加工で対応)
\subsubsection{ハーネス・コネクタ等の電子部品の考慮}
CADでハーネス,コネクタを入れることをお勧めする.ハーネスがどこを通るか,通る場所には通れるだけの隙間があるか,コネクタは干渉しないかを確認する必要がある.また,ハーネスは曲率の限界があるので注意する.
\subsubsection{ハーネスの通り道に突起物を置かない}
特に注意するのが,構造部品の角である.振動等でハーネスの皮膜が破れる恐れがあるため,Rをつける.また,さらに補強としてハーネスにガラスクロステープを巻いた.
\subsubsection{組立を考慮した設計}
組み立てについては,工具の大きさと,組み立てのしやすさを考慮する必要がある.工具の大きさについては,組み立て時に工具と部品に干渉なく組み立てられるかを注意する.底面パネルに対するパッチアンテナ土台の取り付けは,取り付けるねじに対して工具を同軸で取り付けるべきであるが,底面パネルと工具が干渉するので,工具を斜めにして取り付けていた.組み立てのしやすさについては,D-subの取り付け等無理な体勢での組み立てがあったので,良くない.何度かワッシャを構体内に落としたこともあったので注意して欲しい.
\subsubsection{ばか穴による組み立てのズレの考慮}
ばか穴にしている箇所は,ねじの径との差の範囲で部品間が相対的にずれる.0.1mmオーダーの差があり,レール精度は±0.1mmであるので,十分影響がある.要求精度以外にも,ディプロイメントスイッチの接触不良があった.スイッチの位置によってディプロイメントピンとの相対的位置関係が変わり,スイッチ起動の要求,インターフェイスを満たせない場合があった.今回はスイッチ取り付け後のピンを取り付ける際に位置を調整することで対処した.


\subsection{図面作成}
図面作成は寸法を全て書き込めば良いだけでなく,加工する際に必要な寸法を書くことが重要である.自ら加工することも少なくなかったので,そのような経験があるとよりうまく書けると思う.
%%\subsubsection{加工者への考慮}

%%\subsubsection{公差の指定}
\subsubsection{加工順序の考慮}
自分で加工するとき,前日までに加工順序を確認しておくことをお勧めする.当日朝から工場に行ってから工程を考えると進捗が得られない.
\subsubsection{加工者との早めの相談}
公差等わからないことがあったら専門家に聞くほうが良い.考えるより細かい部分は話しながら決めたほうが早く進む.


\subsection{発注}
発注は相手に誤解されないように伝えることが重要である.
\subsubsection{業者と工場の違い}
基本的には工場の発注が良い.工場のメリットは大学内に併設しているので直接コンタクトがとりやすい.外部業者であるとメールのみのやり取りで伝えきれないことがあったりする.外部業者は,工場が立て込んでいて納期が遅かったり,休業のときに利用する.工場は大学内の様々な案件を行っているので,時期によっては納期がだいぶ先になる.また,外部業者で工場が海外であると日本と休日が異なるため,助かることがある.(特にGW)
%%\subsubsection{納期の考慮}


\subsection{組み立て}

\subsubsection{手順の決め方}
基本的には底面パネルから膜展開部に向かって組み立てる.それに加え,左記の流れであると組めない部分は順序を適宜変えた.また,レール精度が求められている部品(底面パネル,側面パネル×4,膜展開部)に関してはトルクはかけずに仮締めを行い,その後精度出しを行った.最終的な手順は「組み立て手順書」に記載する.
\subsubsection{手順書の作り方}
誰でも組み立てができることを目標とした.組み立て時に考えることはないようにする.また,ソフトを書く人にはバス系の組立ができるようにしてもらうことを薦める.ソフトを書き込むために分解が必要なときもあるので,そのたびに構体系の担当がいないと開発が進まない事態になったことが何度もあった.
\subsubsection{精度出しの方法}
精度出しの方法は「組み立て手順書」に記載している.
\subsubsection{基板保持の冶具}
基板保持とCIBの取り付け誤差により,基板がずれて側面パネルが固定できないことが何度かあった.それにより,基板保持とCIBの相対位置を固定する冶具を作成した.
\subsubsection{基板の組み立て誤差}
上記の対応で精度は良くなったが,その先の基板を取り付けた際に再現性がなく,側面パネルと干渉することがあった.これの原因はわかっていない.
\subsubsection{サーミスタの貼付方法}
サーミスタを貼り付ける際は貼り付ける部品にカプトンテープを貼り,その上にサーミスタを置き,上からカプトンテープを貼る.このとき,どちらのカプトンテープも多数の穴を開けた.これは,真空中でカプトンと部品との間の気泡が膨張し,カプトンがはがれることを防ぐためである.
\subsubsection{配線}
ハーネスの経路は,構体部品との干渉を避けることとハーネスの可動域を考慮してどのように動いても 部品間でハーネスをはさまないよう取り回すことを考慮する.後者についてはFMの1回目の振動試験でハーネスを構体部品ではさんでしまっていたので,はさんでしまったハーネスを別のハーネスをくぐらせて動いたとしても挟まれないようにした.
\subsubsection{ディプロイメントスイッチの不具合}
ディプロイメントピンの公差の指定を誤り,底面パネルにピンが固着してしまった.対策としては,底面パネルの穴を削り,微調整した.また,ピンには二硫化モリブデン処理を施し,摩擦係数を減らした.
\subsubsection{工具の管理}
工具は作業が終わっていなくてもその日中には片付けることを徹底する.放置したり,他の人が使ったりした後,翌日に探すことが多々あった.構体系のみならず開発者全員の認識が必要である.


