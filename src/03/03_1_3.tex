\subsection{バッテリ}
バッテリは公称電圧7.6V,放電容量3900mAhのClyde Space社の30Wh Standalone CubeSat Batteryを購入した(図\ref{fig3_1_bat}).バッテリの電気・構造的特性を表\ref{table3_1_bat_spec}に,絶対最大定格を表\ref{table3_1_bat_max}に示す.このバッテリパックは2直3並列のリチウムポリマー電池であり,UN勧告適合品,NASA標準EP-Wi-032適合品である.

\begin{figure}[htbp]
	\begin{center}
		\includegraphics[width=0.5\linewidth]{./03/fig/battery.jpg}
		\caption{30Wh Standalone CubeSat Battery (c)Clyde Space}
		\label{fig3_1_bat}
	\end{center}
\end{figure}

\begin{table}[htbp]
	\begin{center}
		\includegraphics[width=0.9\linewidth]{./03/fig/battery_spec.png}
		\caption{Electrical and Physical Characteristics of Battery \cite{bat_um}}
		\label{table3_1_bat_spec}
	\end{center}
\end{table}


\begin{table}
	\caption{Maximum Ratings of Battery \cite{bat_um}}
	\label{table3_1_bat_max}
	\centering
	\begin{tabular}{ccccc}
		\hline \hline
		\multicolumn{5}{c}{Max Ratings Over Operating Temperature Range (Unless Otherwise Stated)}\\
		&&BCR  & Value  &  Unit  \\
		\hline
		\multirow{3}{*}{Charge Limits}&Voltage&max&8.4&V\\
		&Current&max&6&A\\
		&Current Rate& max &1.53C &Fraction of Capacity\\
		\hline
		\multirow{3}{*}{Discharge Limits}&Voltage&max&6.2&V\\
		&Current&max&6&A\\
		&Current Rate& max &1.53C &Fraction of Capacity\\
		\hline
		\multicolumn{2}{c}{Operating Temperature}&\multicolumn{2}{c}{ -10 to 50} & ${}^\circ$C\\
		\multicolumn{2}{c}{\multirow{3}{*}{Storage Temperature}}&\multicolumn{2}{c}{1 Year: -20 to +20}&\multirow{3}{*}{${}^\circ$C}\\
		&&\multicolumn{2}{c}{3 Months: -20 to +45}&\\
		&&\multicolumn{2}{c}{1 Month: -20 to +60}&\\
		\multicolumn{2}{c}{Vacuum}&\multicolumn{2}{c}{10-5}&torr\\
		\multicolumn{2}{c}{Vibration}&\multicolumn{2}{c}{To [RD-3]}\\
		\hline
	\end{tabular}
\end{table}
	
本バッテリにはセルレベルの過電流,過充電,過放電保護回路(図\ref{fig3-1bat_cpr}),および1並列ごとの過電流保護回路が組み込まれている(図\ref{fig3-1bat_pr}).これらの保護機能については,Clyde Space社から試験報告書を入手し,さらに本衛星開発チームで環境試験(振動,衝撃)前後での充放電特性を測定した(\ref{subsec:bat_test}参照).

また$0^\circ$C以下で自動で動作するヒータが組み込まれており,さらに温度関係なく制御可能な自作ヒータを貼り付けた(図\ref{fig3-1heater}).

FMおよびEMにおいてはバッテリの$\mathrm{I^{2}C}$ラインの故障が生じたために,組み込まれていたテレメトリ取得機能が使用不可能となった.バッテリの電圧,温度等の情報は別系統で取得可能にしていた.
\begin{figure}[htbp]
	\begin{minipage}{0.5\hsize}
		\begin{center}
			\includegraphics[height=0.5\linewidth]{./03/fig/cell_protection.png}
			\caption{Cell Level Protection Circuit Schematic \cite{bat_um}}
			\label{fig3-1bat_cpr}
		\end{center}
	\end{minipage}
	\begin{minipage}{0.5\hsize}
		\begin{center}
			\includegraphics[height=0.5\linewidth]{./03/fig/bat_pr.png}
			\caption{Battery Protection Architecture}
			\label{fig3-1bat_pr}
		\end{center}
	\end{minipage}
\end{figure}

\begin{figure}[htbp]
	\begin{center}
		\includegraphics[width=0.5\linewidth]{./03/fig/heater.jpg}
		\caption{Image of Heater on Battery}
		\label{fig3-1heater}
	\end{center}
\end{figure}
