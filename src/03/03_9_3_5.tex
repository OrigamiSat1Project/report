\subsubsection{球状太陽電池ミッション(サカセ・坂本)}

サカセアドテック社の主導により,スフェラーパワー社(京都府)が開発した球状太陽電池「スフェラー」と三軸織物を組み合わせた球状太陽電池デバイスを開発し,膜上に搭載した.
球状太陽電池の発電により,三軸織物上に搭載されたLEDが光る.このLED光を,伸展カメラ部の写真撮影により確認することを目指した.
伸展カメラ部の写真撮影は,地球の影で実施するため,暗闇でも球状太陽電池が発電するように,膜上デバイス制御部の側面に2つの赤外線LEDを取り付け,この赤外線LEDを光で球状太陽電池が発電し膜上でLEDが光る,という設計とした.
球状太陽電池について詳細には,膜展開部と同じく,文部科学省に提出した報告書に詳しく記載している.

表,裏のどちらの面から光が照射しても発電が可能な宇宙用デバイスの提案を目指し,フライトモデルへの実装を達成した.デバイスを貼付した膜がコンパクトに収納できること,そして地上試験で展開もできることを確認できた.




