\subsection{SAP}
\subsubsection{SAP製作}
OrigamiSat-1ではAZUL社の太陽電池セルを購入し,アレイ化,土台への接着を開発メンバーで行った.アレイ化は,太陽電池セル同士や感光基板をはんだで繋げて,設計どおりの直列数及び形状の太陽電池アレイを作る作業である.アレイ化の詳細手順は「太陽電池セル簡易アレイ化手順書.docx」を参照.
土台への接着は,上記太陽電池アレイを接着剤を用いて,カプトンテープを貼った土台上に接着する作業である.土台への接着の詳細手順は,「Solar\_Cell\_Lay\_Down.pdf」及び「20180607\_OrigamiSat-1\_EM\_SAP接着.docx」を参照.
\subsubsection{SAP試験}
太陽電池の出力特性試験は,太陽シミュレーターを用いて軌道上と同等の照射強度の光を照射し,太陽電池の電圧-電流特性を測定することで,太陽電池が正常に機能しているかを確認するための試験である.試験により測定した出力特性と,太陽電池セルのデータシート記載の出力特性を比較し確認を行った.出力特性試験は太陽電池セル単体,セルのアレイ化後,アレイの土台への接着後,機体組立後に行った.ただし機体組立後については,正しい測定位置に太陽電池面を持ってくることができなかったため簡易的行い,データシートの出力値と大きく変化がないことのみ確認した.組立後に正確に測定をするためには別途専用の治具を製作する必要があると思われる.詳細な試験手順については「ソーラーシミュレーター実験計画書.docx」を参照.また上記の太陽電池出力特性試験とは別に,アレイ化後以降はアレイの一部セルを覆って光が当たらないようにした上で出力を測定し,バイパス機能の確認も行った.


