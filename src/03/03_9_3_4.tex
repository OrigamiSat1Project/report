\subsubsection{SMAアンテナミッション(鳥阪・坂本)}

東京都立大学の鳥阪がこのアドバンストミッションを主導した.宇宙科学研究所の川崎研究室にご支援をいただいた.(このつながりもあって,5.8GHz通信ミッションの検証の目的で川崎研究室の電波暗室を使用させていただいた.)

膜上において,SMAアンテナによるOrigamiSat-1のCW受信と,ひずみゲージを用いた形状計測を目指した.しかし最初にミッションを遂行できる基板がシステム側にもたらされたのがFM膜の最後の収納時であったため,バス部との電気統合を行わないままに膜上に貼付した.膜収納後,FMのMDC基板との電気統合を試みた.しかし,膜上のSMAアンテナ基板と,バス部のMDC基板間で通信ができなかった.この不具合対応をしている時間はなかったため,SMAアンテナからのハーネスを,MDC基板へ結合しなかった.目指したSMAアンテナミッションについて,鳥阪が筆頭著者となり下記ジャーナル論文を出版した.

\noindent A.~Torisaka, et al., ``Development of shape monitoring system using SMA dipole antenna on a deployable membrane structure,'' Acta Astronautica, Vol.~160, July 2019, pp.~147--154.

少なくとも,膜上にSMAアンテナと剛なアンテナ基板を搭載した状態で,膜面の収納・展開を確認できた.この達成により,革新3号機HELIOSの膜上アンテナミッションおよび干渉計ミッションの提案につながった.



