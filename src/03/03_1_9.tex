\subsection{電源系コメント}

\subsubsection{SAP}
\paragraph{SAP製作}
\begin{itemize}
	\item SAP製作及びSAP試験は作業量的には1人いれば十分可能だが,精神衛生上2人以上での作業を勧める.特にSAP出力特性試験では連日3号館101室を真っ暗にして何時間も1人での作業が続き,思っていた以上に負担が大きかった.
	\item SAP製作は非常に時間がかかる上に少しのミスで大きな出戻りが発生する作業である.購入するセル枚数,製作時間には大きくマージンをとっておくことを推奨する.OrigamiSat-1では最終的にFM機体用のセルの予備がなくなり,一部表面にひびが入っているセルを使用することになった.なお,OrigamiSat-1の試験においては,多少表面にひびが入っていても太陽電池の出力に変化はみられなかった.しかし軌道上での耐久性等に影響が出ることも考えられる.製作時間に関しては,アレイ化の作業時間に加えて,接着剤の乾燥時間も必要であるため,1回出戻りが発生すると2週間弱追加でかかった.
	\item 土台への接着後,アレイのうち1つでもセルが破損した場合,基本的にはアレイ全て取り換える必要がある.なお,一部のみの取り換えは不可能ではない.実際OrigamiSat-1では予備のセルがなかったこともあり,開発過程で4直のアレイのうち1枚だけを取り除いて他のセルを再利用したこともある.しかし取り換え作業は非常に困難な上,正常な他のセルも傷つける可能性が高い.
	\item OrigamiSat-1では太陽電池セルの総枚数が18枚であったためSAP製作が可能であったが,今後の衛星開発でもし太陽電池の枚数を増やすことがあれば,予算との兼ね合いもあるが,SAP製作の外注も検討するべきだと思う.SAPの設計次第ではあるが,太陽電池のセル単位の購入ではなくアレイ化までしてもらった状態での購入も選択肢として存在する.BBMからFMまでで合計何枚の太陽電池セルを発注するか,SAPの製作精度,開発時間等を含めSAP製作のどこを自分達で行うか今後の衛星開発では検討することを勧める.
	\item 内之浦でのOrigamiSat-1のE-SSODへの挿入時,IAの点検担当の方から感光基板の端のわずかな浮きを指摘された.この浮きは接着作業が完璧でなかったことが原因で生じたものであり,衛星引き渡し時は,振動等によってこれ以上浮きが進展しないことを説明し理解が得られた.このような浮きが生じないようにするためには,接着作業だけでなく乾燥期間の重し等の状態にも注意が必要である.
\end{itemize}
\paragraph{SAP試験}
\begin{itemize}
	\item 出力特性試験では,あらかじめ大きなブレッドボード上に試験に使う抵抗を一式用意しておき,ハーネスの接続先を変えるだけの状態で試験をした方がよかった.FM用SAP試験では,残り作業量がわずかだと思っていたため毎回抵抗を付け替えて行っていたが,1回の試験での付け替えには時間がかかる上に,最終的には何かと追加試験をやる機会が増えたため,改善しておけばよかった.
	\item 太陽シミュレーター及びサーモパイル(太陽シミュレーター用放射強度計)は非常に高価な機器であるため,取り扱いには特に注意する.また,サーモパイルの管理には注意をすること.一時期サーモパイルが行方不明になっていたようであるが,サーモパイルがなければ正しく放射強度の設定を行うことができず試験にならないため要注意.また,松永研所有の機器であるため,使用許可を必ずとること.使用報告の徹底が行き届いておらず,松永先生に注意されたことがあった.
\end{itemize}