\section{反省会まとめ}


本節では2020/3/23に開催されたOrigamiSat-1反省会についてまとめる.

\subsection{目的}
OrigamiSat-1停波から時間がたち,打ち上げ時に1年以上プロジェクトに関わっていた開発メンバーの多くが卒業する前に,マネジメントチームだけでなく開発メンバーの視点からもでてきた反省点を共有するために,OrigamiSat-1反省会を開催した.
すでに次期衛星の開発が始まっており,OrigamiSat-1で得た教訓を次に活かす目的もあった.

\subsection{方法}
あらかじめGoogleフォームを用いて匿名の調査を行った.延べ23件(7件が良かった点について,残りが反省点)の回答があった.





\subsection{内容}
 引継ぎ・共有資料

- 担当者がいないとわからないことが多かった
- 引継ぎ資料・作業ログがきちんとなかった
- どこで詰まってて、何ができないといけないのかを、担当者に聞かないとわからない状態だった
- 共有資料の種類
- 進捗に関する話
- 専門的技術に関する話
- 細かい技術資料は存在
- どういう思想・経緯でやったかの共有資料が不足
- 不具合管理表を作ろう
- 検索で調べられるように
- 研究室で衛星開発をしてるっていうのは重要なこと
- 柏山さんの引継ぎがよかった
- コードは見ればわかる状態
- その他については資料が作られていた

- 参考:DLASの引継ぎ
- コード見ればわかる状態
- 論文を読めばわかる状態

- 普段からログを作っておくのが理想

ソフトウェア全体進捗

- 問題点
- EM開発完了時のソフトウェア開発が全然だった
- 試験用のソフトウェアはできていたがFM用ではなかった
- コマンド・データハンドリング等のフローチャートはEM時にあるべきだった
- EM終わった段階での目標
- ハードウェアはフライトレベルのものを目指せた
- 要所要所で締め切りがあった
- ソフトは目標が曖昧だった
- 明確な締め切りがなく後回しになっていた
- 最低限のハードができた時点やSILLSの利用で早い段階でやり始めるべき

その他

- 必要以上に徹夜しない
- 必要以上に徹夜しない姿勢はよかったと思う
- 構造2人きりで日々深夜作業はきつかった
- プロマネを明確に決めなかったのが良いか否か
- MTGでは司会を回すのは当事者意識芽生えていいと思う
- CIBの遅れ
- システム自体は冗長性もありよくできていた
- 通信が最後後回しになってしまったのはよくなかった
- 検証期間が短くなってしまった
- 最初にマストのものをやって、あとからミッション部を作っていくべき
- 初期
- 買い物で組み合わせればできるものは後回しになってた
- 伸展部等が優先になってしまってた
- CIBを作ることになった経緯
- インヒビット系が必要
- 持ってるEPSだけだとだめだから、CIB作ることにした
- その他通信のモデム回路等、自作で作るべきものを全部盛り込んでしまった
- 西無線を買った経緯
- 当時はそれしかなかった
- 開発途中でモデム付きのが売られ始めた
- ミッションとバスどっちに力を入れるべきか
- バスをとにかく堅牢にする
- バスを全部買い物にして、ミッションに注力?
- 圧倒的当事者意識
- 学生がミッション作りからもっとやるべきだった?
- 実際にはやっていたが、その頃の学生がいなくなってしまった
- ソフト開発とハード開発の並行して行うべき
- マネジメント不足
- 開発の遅れから、試験前の徹夜につながった
- 締め切り効果に頼らないほうが良かった
- マネジメントはよかったが予期せぬ不具合が発生してしまったby中西
致命的な不具合は開発末期に出てくるので、それを見越して早めのスケジュールで開発を行うべきであった
- CIB機能開発の出だしが遅かった
- 試験用に場当たり的な感じでコードを書いていたのはある
- CW取れれば上出来っていう雰囲気は良くなかった
- 引き渡し直前
- CIBの統合→メモリーオーバー問題
- 週一でやってたMTGではパワポ資料作成の方がいい
- 発表資料ほど作りこまなくても、写真とキャプションくらいはあった方がいい
- 山の頂上からEM電波送信・地上局受信
- 某N大はやってるらしい?
- バス開発だと論文にできないというのはモチベーションが上がらない理由?
- 学会発表:古谷研、飯島さんとかは出してる
- ウカレンには先生方が毎回発表に行ってる
- 全体的に学生の発表回数が少ない
- 古谷研との関わりが薄いのは問題
- アマチュア無線について
- 工学実証ミッションなのでアマチュア無線家との風当たりが強かった
- アマチュア無線家はそういうミッションに対して最近厳しい
- アマチュア無線衛星という主張をし続けたので使わせてもらった
- 実際無線家の人々が多く取ってくれて協力な助っ人だった
- アマチュア帯使うなら、アマチュア無線家としてふるまわないといけない
- 今後もアマチュア帯使うかどうかは考えたほうがいい
- 次にアマチュア帯使わないとしても、しっかりアマチュア無線コミュニティには参加した方が理想
- うまいフェードアウトの仕方を考えるべき
- アマチュア無線との調整に2年くらい使った
- マージンを必要以上に大きくしてしまい、引き渡し時にオーバーした問題
- 引き渡し時の基準は本当に甘々にしとくべき
- BBM時にもっとしっかりしとくべきだった
- 構造
- 3次元計測が大変だったし、それは向こうの最低限の要求以上のこと
- 計測しないと組みたたないという設計・そう応答してしまったのは良くなかった
- 5.8
- アマチュア無線家との関係があったから、やりませんっていうのは言えなかった
- 搭載だけして。。。っていうのが、今思うと衛星全体としては良かったかも

今後

- 報告書を完成させる
- 原因解明→何とか復帰できないか?
- 地上局の維持
- 以下の計画書を提出済み
- Origamiのコマンドを定期的に送る
- CUTEを時々聞く
- 3か月に1回5.8のパラボラ動かす
- OrigamiのOB会をきちんと定期開催したい
