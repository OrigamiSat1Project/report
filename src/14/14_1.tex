\section{反省会まとめ}


本節では2020/3/23に開催されたOrigamiSat-1反省会についてまとめる.

\subsection{目的}
OrigamiSat-1停波から時間がたち,打ち上げ時に1年以上プロジェクトに関わっていた開発メンバーの多くが卒業する前に,マネジメントチームだけでなく開発メンバーの視点からもでてきた反省点を共有するために,OrigamiSat-1反省会を開催した.
すでに次期衛星の開発が始まっており,OrigamiSat-1で得た教訓を次に活かす目的もあった.

\subsection{方法}
あらかじめGoogleフォームを用いて匿名の調査を行った.延べ23件(7件が良かった点について,残りが反省点)の回答があった.





\subsection{課題・改善点}
議事録を基に当日の議論内容を箇条書きで以下にまとめた.

\subsubsection*{引継ぎ・共有資料}
\begin{itemize}
	\item[課題] 担当者がいないとわからないことが多かった
	\begin{itemize}
		\item 引継ぎ資料・作業ログがきちんとなかった
		\item どこで詰まってて、何ができないといけないのかを、担当者に聞かないとわからない状態だった
		\item 共有資料の種類
		\begin{itemize}
			\item 進捗・細かい技術資料は存在
			\item どういう思想・経緯でやったかの共有資料が不足
		\end{itemize}
		
		\item 良い引継ぎとは
		\begin{itemize}
			\item ソフトウェアはコードを読めば理解できる状態
			\item 論文を読めばわかる状態
			\item その他は引継ぎ資料を作成
		\end{itemize}		
	\end{itemize}
	\item[解決策] 不具合管理の徹底,適度な資料作成
		\begin{itemize}
			\item 普段から作業ログを作成
			\item 不具合は検索可能な状態にする
			\item 週一でやってたMTGではパワポ資料作成の方がいい
			\item 発表資料ほど作りこまなくても、写真とキャプションくらいはあった方がいい
		\end{itemize}

\end{itemize}


\subsubsection*{ソフトウェア全体進捗}
\begin{itemize}
	\item[課題] EM開発完了時のソフトウェアが非常に完成度の低いものだった
	\begin{itemize}
		\item 試験用のソフトウェアはできていたがFM用ではなかった
		\item 試験用に場当たり的な感じでコードを書いていたのはある
		\item コマンド・データハンドリング等のフローチャートはEM時にあるべきだった
		\item EM終わった段階での目標
		\begin{itemize}
			\item 要所要所で締め切りがあったハードウェアはフライトレベルのものを目指せた
			\item ソフトは目標が曖昧で明確な締め切りがなく後回しになっていた
		\end{itemize}
	\end{itemize}
	\item[解決策] 最低限のハードができた時点やSILLSの利用で早い段階で開発を始めるべき
	
\end{itemize}

\subsubsection*{CIB開発}
\begin{itemize}
	\item 開発が遅れ検証が期間が短くなってしまった
	\begin{itemize}
		\item システム自体は冗長性もありよくできていた
		\item 通信が最後後回しになってしまったのはよくなかった
		\item ミッションとバスどっちに力を入れるべきか
		\begin{itemize}
			\item 最初にマストのものをやって、あとからミッション部を作っていくべき
			\item バスをとにかく堅牢にする
			\item バスを全部買い物にして、ミッションに注力?
		\end{itemize}
	\end{itemize}
	\item[解決策] 最低限のハードができた時点やSILLSの利用で早い段階で開発を始めるべき
\end{itemize}


\subsubsection*{マネジメント}
\begin{itemize}
	\item 引渡し前,必要以上に徹夜しない姿勢はよかったと思う
	\item 構造2人きりで日々深夜作業はきつかった
	\item 最後学生プロマネを明確に決めなかったのが良いか否か
	\begin{itemize}
		\item MTGでは司会を回すのは当事者意識芽生えていいと思う
	\end{itemize}
	\item ソフト開発とハード開発の並行して行うべき
	\item 開発の遅れから、試験前の徹夜につながった
	\item 締め切り効果に頼らないほうが良かった
	\item 致命的な不具合は開発末期に出てくるので、それを見越して早めのスケジュールで開発を行うべきであった
\end{itemize}



\subsubsection*{モチベーション}
\begin{itemize}
	\item 圧倒的当事者意識
	\begin{itemize}
		\item 学生がミッション作りからもっとやるべきだった?
		\item 実際にはやっていたが、その頃の学生がいなくなってしまった
	\end{itemize}
	
	\item CW取れれば上出来っていう雰囲気は良くなかった
	\item バス開発だと論文にできないというのはモチベーションが上がらない理由?
	\begin{itemize}
		\item 学会発表:ミッション系はしている
		\item 宇科連には先生方が毎回発表に行ってる
		\item 全体的に学生の発表回数が少ない
	\end{itemize}
	\item ミッション系開発チームとバス系開発チームのかかわりが薄かったのは問題
\end{itemize}


\subsubsection*{その他}
\begin{itemize}
	\item 電圧に関する安全基準のマージンを必要以上に大きくしてしまい、引き渡し時にオーバーしたため,引き渡し時の基準は本当に甘々にしとくべき
	\item 構体の3次元計測が大変だったし、それは向こうの最低限の要求以上のこと
	\item 計測しないと組みたたないという設計・そう応答してしまったのは良くなかった
\end{itemize}






\subsection{コメント}
Googleフォーム回答の皆のコメントを順不同で,原文通り以下に記述した.
\begin{itemize}
	\item 初めてのことに対してどれくらい時間がかかるか見積もるのは面倒で難しいですが、その見積もりに対して3倍余裕を持たせておくと予期せぬエラーに対応できると思います。
	\item やっぱりノウハウがほとんどない状態での開発はしんどかったし,開発期間が長すぎてモチベーションも全体的に低かったと思うので,短いスパンで衛星開発に取り組もうとしている現在の坂本研の流れはいい感じだと思います!
	\item 開発メンバーの中で,複数の学年にまたがる何人かが当事者意識が一番大事だと言っていた.この当事者意識が芽生えた人は良くプロジェクトに貢献していたと思う.逆に言えば,できる限り多くのメンバーをできる限り早い段階でこの状態に持っていくことが衛星開発の一つの鍵であり,OrigamiSat-1に関してはこの状態になるのに時間がかかったと思う.
	\item 衛星開発に携わり身体的・精神的に大変なことも多々あったのは事実であるが,一方で自分の手にしたものが今も地球を周回しているという事実にはある種の達成感を覚える.また同期とはこのプロジェクト(とCanSat)を通じて仲良くなれた.いまだに飲み会ではそれらプロジェクトの珍話で盛り上がる.後輩には各々の無理のない範囲,興味が持てる範囲で体調に気を付けながら精一杯開発に勤しんでもらいたい.
	\item OrigamiSat-1開発に関わった皆様、大変お疲れ様でした。短命ではあったものの、軌道上での動作が確認できたことは、非常に良かったと思います。次の開発では是非サクセスクライテリア全達成してほしいです。
	\item 本来引き渡し1週間前はソフトウェア開発なしにテストのみ行うべきである.
	\item 完全に開発当初の知識不足が,統合時の混乱を招いたと感じています.EM/FM開発に関わった学生のがんばりは驚異的ですばらしかったので,そのがんばりをミッション成功につなげられなかったマネジメントを申し訳なく思っています.
	\item 日本国のロケットに載せてもらうという機会をいただき,皆で非常に濃密な体験ができたことを幸運と考え,この経験をもとに次は何をしてやろう?とワクワクして次のステップへ進めればと思います.徹夜はほどほどにして,体には気をつけましょう.
\end{itemize}