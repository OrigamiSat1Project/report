\section{開発日程}

本章では本衛星の開発日程について述べる.



\subsection{BBM開発}

2015.1-2018.8

本衛星はバス部BBM開発に非常に多くの時間が費やされた.衛星を作ること,特にバス部のノウハウが失われた状態で作ることの困難さに対して開発初期は誰もが楽観視していたため


MDR 2016.3



SDR 2016.8

当初SDRを行うことは想定されておらず同時期にPDRを行う予定であった.
しかし開発の遅れから


2016年秋冬以降
統合試験が開始されたが
開始されはじめてFM通信がコンポーネント開発レベルから抜けていないことが明らかになった

安全審査 Phase 0/I




PDR 2017.8
FM信号送受信を含む基本的なすべての機能確認が行われ,PDR開催に至った



\subsection{EM開発}

PDR後の統合試験


2018年9月から新たに学生プロジェクトマネージャーという立場が設けられた.これは先生の要求にただ答えるという開発体制では学生のモチベーションが上がらず,開発が完了しないことが明らかであったためである.

その後環境試験等のマイルストーンの可視化が行われ

Google Driveに皆がすぐアクセスできるスケジュール表が作られ,毎週のミーティングでスケジュール確認から行うようになった


CDR
CDRの日程が後ろ倒しになり続けていたため
EM開発未完了のまま

当時残っていた問題は


安全審査 Phase II

\subsection{FM開発}

FM振動試験

FM再振動試験

FM熱真空試験

安全審査 Phase III

衛星引き渡し