\section{開発の目的・ミッション定義}

\subsection{開発の目的}

1. 多機能展開膜構造の宇宙実証と数値解析モデル高精度化

2. 継続的な展開構造物 宇宙実証のための「実験プラットフォーム」構築

3. 大学・企業が連携し宇宙実証を行う「研究・開発拠点」形成

\subsection{ミッション定義}

上述3つの目的達成のためのパイロットプロジェクトとして,以下の3つのミッションを実施する.

多機能膜展開ミッション: 今後多様なアプリケーションへの発展を実現する目的で,開発者らが提案する多機能膜構造の展開・展張特性を軌道上で評価し,(i) 膜構造の設計・検証手法の発展に寄与するデータを取得するとともに,(ii) 多様なユーザーへアピールする.

宇宙実証プラットフォーム開発ミッション: 先進的展開構造の宇宙実証を容易にする目的で,主に大学・企業の研究者・技術者らに向けた宇宙実証プラットフォームを構築する.具体的には,(i) 市販衛星部品を利用した衛星開発を行いながら、(ii) 伸展マストを用いた軌道上撮影技術を獲得する.

アマチュア高速通信ミッション: 衛星通信技術の発展と普及を目的に、アマチュア無線5.8GHz帯を用いた高速通信を実現する。

より詳細な記述は以下である.

(M1)	多機能膜展開ミッション: 10cm × 10cm の筐体内から 1m × 1m サイズへと 2 次元方向に展開する多機能展開膜を軌道上で展開し、展開挙動・展張状態のデータを取得する。また、膜には薄膜デバイス(形状記憶合金アンテナ、太陽電池、球状太陽電池)を添付し、これらが機能することを検証する。

(M2)	宇宙実証プラットフォーム開発ミッション: 今後の実証実験を容易にするため、(A) 販売機器を利用した衛星開発を行い、かつ (B) 伸展マストを用いた軌道上撮影技術を開発して、多機能展開膜の展開挙動および展張形状を計測する。

(M3)	高速通信ミッション: アマチュア無線 5.84GHz 高速通信を用いた衛星通信を実現する。福岡工業大学が 2013 年に FITSAT-1 を用いて開発した衛星通信技術を実装・運用する。

OrigamiSat-1の概念図を図 1(a)に示す。上記ミッション(M1)に述べた多機能展開膜を図 1(b)と(c)にそれぞれ示す。図 1(b)は初期のエンジニアリングモデル(EM)、図 1(c)はフライトモデル(FM)である。EMのダミーデバイス薄膜は褐色である一方、FMにおいては透明に近いダミーデバイスを展開膜の全面に貼付している。FMにおいてはステレオ撮影のためのターゲットマーカも貼付されている。また上記ミッション(M2)-(B) に述べた、伸展マストを用いた展開構造の画像計測の概念図を図 2に示す。

\subsection{サクセスクライテリア}
図で示す.

\subsection{ミッションシークエンス}
図 8に軌道投入後のミッションシークエンスを示す。本衛星はロケットより放出後、VHFおよびUHF用のモノポールアンテナを展開し、初期チェックアウトを実施する。その後、伸展カメラ部のマストを1m伸展することにより膜展開部がカメラの画角に入るよう調整し、多機能膜の展開・計測を行う。膜展開後は、膜形状の経時変化を定期的にカメラで計測する他、膜上の薄膜デバイスの機能検証を実施する。軌道高度が400km付近まで低下した後は軌道寿命を延長するため、伸展マストおよび膜展開部を切り離し2UサイズCubeSatとして、主に5.8GHz帯を用いた高速通信実験および、撮影データのダウンリンクを行う。

