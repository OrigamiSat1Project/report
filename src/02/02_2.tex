\section{システム要求}


\subsection{ミッション系}

サクセスクライテリアに対応させ,ミッション系に以下を要求した.

\subsubsection{ミニマムサクセス}

\begin{enumerate}
\renewcommand{\labelenumi}{(M\arabic{enumi})}
\renewcommand{\labelenumii}{(mR-\arabic{enumii})}
\item バス確認
\begin{enumerate}
\item 衛星から地上局へダウンリンクができること
\item 地上局から衛星へアップリンクができること
\end{enumerate}
\item 5.8GHz実証
\begin{enumerate}
	\setcounter{enumii}{2}
\item アマチュア5.8GHz帯により1024×768 pixel画像をダウンリンクできること\footnote{(注)福岡工大FITSAT-1は640×480を使用}
\end{enumerate}
\item マスト伸展開始
\begin{enumerate}
		\setcounter{enumii}{3}
\item 伸展マストの伸展が開始できること
\item 伸展マストの伸展長を推定できるエンコーダデータが得られること
\item 伸展マストの伸展長を推定できる画像が得られること
\end{enumerate}
\item 膜展開開始
\begin{enumerate}
		\setcounter{enumii}{6}
\item 多機能膜の展開が開始できること
\item 展開機構の動作が確認できること
\item 多機能膜展開開始がわかる角速度あるいは加速度データが得られること
\item 多機能膜展開がわかるTLEデータが得られること
\item 多機能膜展開過程あるいは展開後の画像が得られること
\end{enumerate}
\end{enumerate}

\subsubsection{フルサクセス}

\begin{enumerate}
	\renewcommand{\labelenumi}{(F\arabic{enumi})}
	\renewcommand{\labelenumii}{(mR-\arabic{enumii})}
\item マスト伸展完了
\begin{enumerate}	
			\setcounter{enumii}{11}
	\item 伸展マストを正常に伸展完了できること
	\item 伸展マストの伸展によって膜が全展開した場合に膜全体が撮影可能な位置へ膜展開部を移動したことを確認できる画像,あるいはエンコーダデータおよび画像,が得られること
\end{enumerate}
\item 膜展開完了
\begin{enumerate}	
				\setcounter{enumii}{13}
	\item 多機能膜の投影面積 90\%以上を展開できること
	\item 膜面積90\% 以上の展開完了を確認できる画像が得られること
\end{enumerate}
\item 展開ダイナミクス計測
\begin{enumerate}	
				\setcounter{enumii}{15}
	\item 展開前1枚,展開後1枚を含み,多機能膜展開過程でタイムスタンプのついた画像5枚以上が得られること
	\item 多機能膜展開挙動について知見が得られるサンプリングレート(160Hz以上)で,展開過程において角速度データが得られること
	\item 多機能膜展開挙動について知見が得られるサンプリングレート(角速度計と同一レート)で,展開過程において加速度データが得られること
\end{enumerate}
\end{enumerate}

\subsubsection{アドバンストサクセス}
\begin{enumerate}
	\renewcommand{\labelenumi}{(A\arabic{enumi})}
	\renewcommand{\labelenumii}{(mR-\arabic{enumii})}
	\item 膜形状計測
\begin{enumerate}	
				\setcounter{enumii}{18}
	\item 多機能膜の展開後の面外形状モードを推定できるステレオ画像(面外方向の標準偏差±10mm以内)が得られること
\end{enumerate}
\item 薄膜太陽電池
\begin{enumerate}	
				\setcounter{enumii}{19}
	\item 多機能膜展開後に薄膜太陽電池のI-V特性と温度,および本体の各ソーラーパネルの発電量が得られること
\end{enumerate}
\item 膜上アンテナ
\begin{enumerate}	
				\setcounter{enumii}{20}
	\item 多機能膜展開前後でSMAアンテナ受信ゲインの変化が確認できること
\end{enumerate}
\item SMA形状制御
\begin{enumerate}	
				\setcounter{enumii}{21}
	\item 多機能膜展開前後でSMAアンテナ伸展を確認できるひずみゲージデータを取得できること
\end{enumerate}
\item 球状太陽電池
\begin{enumerate}	
				\setcounter{enumii}{22}
	\item 多機能膜展開後に球状太陽電池による膜上LED発光を確認できる画像を取得できること
\end{enumerate}
\item 動画取得
\begin{enumerate}	
				\setcounter{enumii}{23}
	\item 40fps以上のレートの膜展開動画が得られること
\end{enumerate}
\item 膜経時変化
\begin{enumerate}	
				\setcounter{enumii}{24}
	\item 軌道1周期(約90分)以上経過後に再度,多機能膜の展開後の面外形状モードを推定できるステレオ画像(mR-18と同一精度)が得られること
\end{enumerate}
\item 膜切り離し
\begin{enumerate}	
				\setcounter{enumii}{25}
	\item 展開膜全体をバス部から切り離せること
\end{enumerate}
\item 側面カメラ
\begin{enumerate}	
				\setcounter{enumii}{26}
	\item 伸展マストが伸展完了しない場合を想定しても膜展開が確認できる画像が得られること
\end{enumerate}
\end{enumerate}






\subsection{バス系}
バス系に対するシステム要求を表 \ref{requirment_bus} に示す.

\begin{table}[htb]
    \centering
    \caption{バス系システム要求一覧}
    \begin{tabular}{|c|c|p{11cm}|} \hline
        項目 & 要求番号 & 要求 \\ \hline
        \multirow{5}{*}{通信} & bR-1 & アマチュア無線UHF帯のFMおよびCWダウンリンク機能を有すること.
        \\ \cline{2-3}
        & bR-2 & アマチュア無線VHF帯のFMアップリンク機能を有すること.
        \\ \cline{2-3}
        & bR-3 & アマチュア5.8GHz帯のFM送信機能を有すること.\\ \cline{2-3}
         & bR-4 & 地球局からのコマンドで,宇宙機からの電波放射を停止で
        きること. \\ \cline{2-3}
        & bR-5 & 国際周波数調整により定められた電波の型式,周波数で通
        信を行えること. \\ \hline
         \multirow{4}{*}{電源} & bR-6 & 太陽電池により必要な電力を獲得
         できること.\\ \cline{2-3}
        & bR-7 & 太陽電池で発電した電力をバッテリに蓄えることが出来る
         こと.  \\ \cline{2-3}
        & bR-8 & バッテリ電力を各コンポーネントに供給できること.\\ \cline{2-3}
         & bR-9 & 打上時は電力供給が遮断され,放出後に供給が開始される
         こと. \\ \hline
         \multirow{5}{*}{C \& DH} & bR-10 & 地球局からのコマンドに基づ
         き,通信機,EPS, 伸展カメラ部,膜上デバイス制御部,膜展開部を
         制御する機能を有すること.\\ \cline{2-3}
        & bR-11 & 通信機,EPS, 伸展カメラ部,膜上デバイス制御部,膜展
         開部からのデータを収集し,テレメトリを生成できること.\\ \cline{2-3}
        & bR-12 & 取得データおよび,テレメトリを蓄えることができること.\\ \cline{2-3}
         & bR-13 & 取得データおよび,テレメトリを通信機を介して地上に
         ダウンリンクできること.\\ \cline{2-3}
         & bR-14 & 地上からのコマンドにより再起動できること.\\ \hline
         \multirow{2}{*}{構造} & bR-15 & 打上時,放出時の振動・衝撃で
         破損しないこと.\\ \cline{2-3}
        & bR-16 & 打上時,可動部は HRM (Hold and Release Mechanism) に
         よって固定され,放出後定められたタイミングで解放されること.
         \\ \hline
         熱 & bR-17 & 打上時,及び軌道上における温度環境で機能喪失しな
         いこと.\\ \hline
         放射線 & bR-18 & 各機器は軌道上の放射線環境下において,ミッショ
         ン終了まで破損しないこと.\\ \hline
    \end{tabular}
    \label{requirment_bus}
\end{table}

\subsection{ロケットインターフェース}

ロケットとのインターフェースは,JX-ESPC-101655 「OrigamiSat-1 \/ イプシ
ロンロケット4号機インタフェース管理文書」にて定義されている.これを満
たすための要求を表 \ref{requirment_interface} に示す.

\begin{table}[htb]
    \centering
    \caption{ロケットインターフェースに関連する要求一覧}
    \begin{tabular}{|p{1.7cm}|c|p{10cm}|} \hline
        要求番号 & 源泉文書項目 & 要求 \\ \hline
        4.2項 & iR-1 & 衛星の最低次固有振動数は、レール4本の両端部を固
        定した条件で、113Hz以上であること. \\ \hline
        4.3項 & iR-2 & 規定された外形寸法を満たすこと. \\ \hline
        4.4項 & iR-3 & 衛星 $Z_{sc}$ 軸に平行な四辺に,規定された長さ,
        幅,形状,表面処理のレールを有すること. \\ \hline
        4.5項 & iR-4 & 衛星の主構体及び突起部を規定されたエンベロープ
        内に収めること.
        \\ \hline
        4.6項 & iR-5 & 質量は4.5kg以下であること,また,質量中心は4本
        のレールで構成される直方体の幾何中心を中心とする半径20mmの球体
        内に位置すること. \\ \hline
        4.7(2)項,5.1(1)-(5)項 & iR-6 & 所定の形状・機能のディプロイメ
        ントスイッチを所定の位置に搭載すること.\\ \hline
        4.8項 & iR-7 & 地上での取り扱い,試験,運搬,打ち上げ,運用等,
        全ての環境において破損や永久変形することなく,材料の許容応力に
        対して安全余裕(MS)が正となること. \\ \hline
        4.9項 & iR-8 & E-SSODへの収納後に外部からアクセスするポートは,
        定められたアクセス窓の位置に配置すること.\\ \hline
        5.1(6)項 & iR-9 & 太陽電池及びバッテリ電力による衛星起動に対し
        て,ディプロイメントスイッチを含めて,3つ以上,電力を遮断する
        手段を設けること.\\ \hline
        5.2項 & iR-10 & E-SSOD収納後に地上での取り扱いが必要になった場
        合に備え,衛星はアクセス窓側にボンディングポイントを有すること.
        \\ \hline
        6項 & iR-11 & 衛星はコールドロンチとする.射場作業開始~ロケッ
        ト・衛星分離後200s経過するまでRF放射しないこと.また,定められ
        た許容値を超える磁気を放射しないこと.\\ \hline
        7.4.1(2)項 & iR-12 & CubeSat放射後200秒経過するまで展開物を展開しないこと.
        \\ \hline
        8.5.1項 & iR-13 & 定められた,ロケット及び射場からの放射電界レ
        ベルを許容すること.\\ \hline
        9項 & iR-14 & 定められた機械環境適合試験(正弦波振動,ランダム
          振動,モーダルサーベイ,衝撃,質量特性)を実施すること.
        \\ \hline
    \end{tabular}
    \label{requirment_interface}
\end{table}

\subsection{システム安全}
本衛星のシステム安全については,以下の文書(ハザードレポート)により定
義・管理を実施した.
\begin{itemize}
  \item OP-S1-0041 OrigamiSat-1 スタンダードハザードレポート(STD-HR)
  \item OP-S1-0042 OrigamiSat-1 UHR-1(衛星の構造破壊)
  \item OP-S1-0043 OrigamiSat-1 UHR-2(展開機構の誤展開)
  \item OP-S1-0050 OrigamiSat-1 UHR-3(意図しない電波放射)
  \item OP-S1-0053 OrigamiSat-1 UHR-4(バッテリの破裂・発火)
\end{itemize}
これらの文書からの設計要求を表\ref{requirment_safety1}および表\ref{requirment_safety2}に示す.
上記文書内,OP-S1-0042, OP-S1-0043 については,システム安全審査 Phase
3 の際に,識別すべきハザードから除外されたが,開発完了時であるため,設
計要求については,設計時の物を記載する.

\begin{table}[htb]
    \centering
    \caption{システム安全に関連する要求一覧 1}
    \begin{tabular}{|p{2cm}|c|c|p{8cm}|} \hline
        対応ハザードレポート& 項目番号 & 要求番号 & 要求 \\ \hline
        \multirow{6}{*}{STD-HR} & 1.1 b)-1 & hR-1 & ロケット搭載時にお
        ける通常の運用コンフィギュレーションにおいて(故障は想定する必
          要はない),自身または混載ペイロードの推進薬のリークに対して
        点火源とならない.\\ \cline{2-4}
        & 1.1 b)-3項 & hR-2 & 振動試験/衝撃試験により電子機器の健全性
        を確認する.(バッテリ直下のリレーが非密封性の場合は、チャタリングの確認を含む)
        \\ \cline{2-4}
        & 1.1 b)-4項 & hR-3 & 適切に接地することで静電気を防止する.\\ \cline{2-4}
        & 1.1 b)-5項 & hR-4 & ヒドラジンに対する触媒作用となる錆などが
        ないことを管理する.(防錆材料の使用) \\ \cline{2-4}
        & 2 b)項 & hR-5 & 電波放射は、3インヒビットスイッチによって発
        生しないようにコントロールする.\\ \cline{2-4}
        & 9 b)項 & hR-6 & 露出した高温、低温表面がない設計とする.
        \\ \hline
        UHR-1 & 1.1項 & hR-7 & 規定の安全係数(降伏荷重に対して1.25、終
        極荷重に対して1.5(TBD))を掛けた荷重を印加された時の安全余裕が
        正となる構造とする.\\ \hline 
        \multirow{7}{*}{UHR-2} & 1.2項 & hR-8 & 展開力及び,打上時の振動に対し
        て十分な強度を有する保持解放機構を有すること.\\ \cline{2-4}
        & 2.3項 & hR-9 & 収納時に発生する張力と打上げ時の機械環境によ
        り発生する荷重に対して、充分な強度を持つテグスを選定する.
        \\ \cline{2-4}
        & 2.4項 & hR-10 & テグスの結び目にほつれが生じない結び方を選定
        する.\\ \cline{2-4}
        & 2.5項 & hR-11 & 収納時に発生する張力を長期間受けることで生じ
        る伸び量を吸収できるテグスの引き回し方法・張力を選定する.
        \\ \cline{2-4}
         & 2.7項 & hR-12 & 収納時に発生する張力と打上げ時の機械環境に
        より発生する荷重に対して,充分な強度を持つようテグス基部(ボビ
          ンフランジ)を設計し,テグスの緩み発生を防止する.
        \\ \cline{2-4}
        & 3.1項 & hR-13 & ニクロム線へ電源を供給する回路に放出検知ピン
        による3インヒビットを設け,放出ポッド搭載から放出までの電源供
        給を遮断する.\\ \cline{2-4}
        & 3.2項 & hR-14 &ニクロム線へ電源を供給する回路にフライトピン
        による3インヒビットを設け,放出ポッド格納までの電源供給を遮断
        する.\\ \hline
    \end{tabular}
    \label{requirment_safety1}
\end{table}

\begin{table}[htb]
    \centering
    \caption{システム安全に関連する要求一覧 2}
    \begin{tabular}{|p{2cm}|c|c|p{8cm}|} \hline
        対応ハザードレポート& 項目番号 & 要求番号 & 要求 \\ \hline
        \multirow{2}{*}{UHR-3} & 1.1項 & hR-15 & 通信系へ電源を供給す
        る回路に放出検知ピンによる3インヒビットを設け,放出ポッド搭載
        から放出までの電源供給を遮断する.\\ \cline{2-4}
        & 1.2項 & hR-16 &通信系へ電源を供給する回路にフライトピン(2ヶ
          所)を設け,放出ポッド搭載までの電源供給を遮断する.
        \\ \hline
        \multirow{6}{*}{UHR-4} & 1.1項 & hR-17 & バッテリを充電する回
        路に放出検知ピンによる3インヒビットを設け,放出ポッド搭載から
        放出までの電源供給を遮断する.\\ \cline{2-4}
        & 1.2項 & hR-18 & バッテリを充電する回路にフライトピンを設け,
        放出ポッド搭載までの電源供給を遮断する.\\ \cline{2-4}
        & 2.2項 & hR-19 & 環境試験前後で充放電特性に変化のないバッテリを使用する.
        \\ \cline{2-4}
        & 4.1項 & hR-20 & 打上環境に耐性のあるバッテリセルを選定する.
        \\ \cline{2-4}
        & 4.2項 & hR-21 & バッテリ搭載時にセルに機械的ストレスが生じな
        い取付構造の設計とする.\\ \cline{2-4}
        & 4.3項 & hR-22 & 引渡しから放出までの間,バッテリが保管温度範
        囲を逸脱しない設計とする.\\ \hline
    \end{tabular}
    \label{requirment_safety2}
\end{table}

\subsection{国際宇宙ステーションへのリスク防止要求}
本衛星の運用にあたっては,国際宇宙ステーションに対するリスクを防止する
観点から,JAXAと協議の上,表\ref{requirment_op}に示す要求を設定した.

\begin{table}[htb]
    \centering
    \caption{国際宇宙ステーションへのリスク防止に関する要求一覧}
    \begin{tabular}{|c|p{12cm}|} \hline
        要求番号 & 要求 \\ \hline
        oR-1 & ISSより高い軌道で膜を分離する場合,ISSの軌道を横切るタイミングでISSと200m/s以上の相対速度があること.\\ \hline
        oR-2 & ISSの軌道範囲に入る4日前からISSのPerigeeを抜けるまでの間は分離や膜展開など物体数やBN(Ballistic Number)が大きく変わるようなアクションを実施しないこと.\\ \hline
    \end{tabular}
    \label{requirment_op}
\end{table}
