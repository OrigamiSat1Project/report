\subsection{ミッション系}

サクセスクライテリアに対応させ,ミッション系に以下を要求した.

\subsubsection{ミニマムサクセス}

\begin{enumerate}
\renewcommand{\labelenumi}{(M\arabic{enumi})}
\renewcommand{\labelenumii}{(mR-\arabic{enumii})}
\item バス確認
\begin{enumerate}
\item 衛星から地上局へダウンリンクができること
\item 地上局から衛星へアップリンクができること
\end{enumerate}
\item 5.8GHz実証
\begin{enumerate}
	\setcounter{enumii}{2}
\item アマチュア5.8GHz帯により1024×768 pixel画像をダウンリンクできること\footnote{(注)福岡工大FITSAT-1は640×480を使用}
\end{enumerate}
\item マスト伸展開始
\begin{enumerate}
		\setcounter{enumii}{3}
\item 伸展マストの伸展が開始できること
\item 伸展マストの伸展長を推定できるエンコーダデータが得られること
\item 伸展マストの伸展長を推定できる画像が得られること
\end{enumerate}
\item 膜展開開始
\begin{enumerate}
		\setcounter{enumii}{6}
\item 多機能膜の展開が開始できること
\item 展開機構の動作が確認できること
\item 多機能膜展開開始がわかる角速度あるいは加速度データが得られること
\item 多機能膜展開がわかるTLEデータが得られること
\item 多機能膜展開過程あるいは展開後の画像が得られること
\end{enumerate}
\end{enumerate}

\subsubsection{フルサクセス}

\begin{enumerate}
	\renewcommand{\labelenumi}{(F\arabic{enumi})}
	\renewcommand{\labelenumii}{(mR-\arabic{enumii})}
\item マスト伸展完了
\begin{enumerate}	
			\setcounter{enumii}{11}
	\item 伸展マストを正常に伸展完了できること
	\item 伸展マストの伸展によって膜が全展開した場合に膜全体が撮影可能な位置へ膜展開部を移動したことを確認できる画像,あるいはエンコーダデータおよび画像,が得られること
\end{enumerate}
\item 膜展開完了
\begin{enumerate}	
				\setcounter{enumii}{13}
	\item 多機能膜の投影面積 90\%以上を展開できること
	\item 膜面積90\% 以上の展開完了を確認できる画像が得られること
\end{enumerate}
\item 展開ダイナミクス計測
\begin{enumerate}	
				\setcounter{enumii}{15}
	\item 展開前1枚,展開後1枚を含み,多機能膜展開過程でタイムスタンプのついた画像5枚以上が得られること
	\item 多機能膜展開挙動について知見が得られるサンプリングレート(160Hz以上)で,展開過程において角速度データが得られること
	\item 多機能膜展開挙動について知見が得られるサンプリングレート(角速度計と同一レート)で,展開過程において加速度データが得られること
\end{enumerate}
\end{enumerate}

\subsubsection{アドバンストサクセス}
\begin{enumerate}
	\renewcommand{\labelenumi}{(A\arabic{enumi})}
	\renewcommand{\labelenumii}{(mR-\arabic{enumii})}
	\item 膜形状計測
\begin{enumerate}	
				\setcounter{enumii}{18}
	\item 多機能膜の展開後の面外形状モードを推定できるステレオ画像(面外方向の標準偏差±10mm以内)が得られること
\end{enumerate}
\item 薄膜太陽電池
\begin{enumerate}	
				\setcounter{enumii}{19}
	\item 多機能膜展開後に薄膜太陽電池のI-V特性と温度,および本体の各ソーラーパネルの発電量が得られること
\end{enumerate}
\item 膜上アンテナ
\begin{enumerate}	
				\setcounter{enumii}{20}
	\item 多機能膜展開前後でSMAアンテナ受信ゲインの変化が確認できること
\end{enumerate}
\item SMA形状制御
\begin{enumerate}	
				\setcounter{enumii}{21}
	\item 多機能膜展開前後でSMAアンテナ伸展を確認できるひずみゲージデータを取得できること
\end{enumerate}
\item 球状太陽電池
\begin{enumerate}	
				\setcounter{enumii}{22}
	\item 多機能膜展開後に球状太陽電池による膜上LED発光を確認できる画像を取得できること
\end{enumerate}
\item 動画取得
\begin{enumerate}	
				\setcounter{enumii}{23}
	\item 40fps以上のレートの膜展開動画が得られること
\end{enumerate}
\item 膜経時変化
\begin{enumerate}	
				\setcounter{enumii}{24}
	\item 軌道1周期(約90分)以上経過後に再度,多機能膜の展開後の面外形状モードを推定できるステレオ画像(mR-18と同一精度)が得られること
\end{enumerate}
\item 膜切り離し
\begin{enumerate}	
				\setcounter{enumii}{25}
	\item 展開膜全体をバス部から切り離せること
\end{enumerate}
\item 側面カメラ
\begin{enumerate}	
				\setcounter{enumii}{26}
	\item 伸展マストが伸展完了しない場合を想定しても膜展開が確認できる画像が得られること
\end{enumerate}
\end{enumerate}



