\section{Phase II}

安全審査 Phase II では,EM検証試験結果までのシステム安全検証の妥当性に
ついて評価を受けた.プロジェクトメンバーからは,坂本,中西,池谷が会場
で参加し,他のメンバーについては Polycom 遠隔会議システムにより東工大
より参加した.

\subsection{Phase 0/Iからの変更点}

本フェーズでは,開発の進展および設計の見直しにより,Phase 0/I から,
以下のいくつかの変更を行った.

\begin{description}
  \item[全般] 電源インヒビットに用いる放出検知スイッチをメカニカルスイッチ
  による直接遮断からメカニカルスイッチによる
  Photo MOS リレー制御へ変更.電流遮断は Photo MOS で行う.(メカニ
    カルスイッチへの電流負荷が過大となった為)
  \item[JMR-003C要求適合マトリクス] 膜展開機構において,溶断後にワイヤ
    片が放出されることを避けるため,FMでは溶断方法をニクロム線2本によ
    る同時加熱から,一本ずつの加熱へ変更した.
  \item[JMR-003C要求適合マトリクス] 膜展開部を衛星から分離するに当たり,
    分離物が軌道上に残存し,衛星に再衝突する恐れが無い事,および,JAXA
    殿より提示された条件 <1. ISSより高い軌道で膜を分離する場合,ISSの軌
    道を横切るタイミングでISSと200m/s以上の相対速度があること.2. ISS
    の起動範囲に入る4日前からISSのPerigeeを抜けるまでの間は分離や膜展
    開等物体数やBN (Ballistic Number) が大きく変わるようなアクションを
    実施しないこと>を順守する運用を行う事を追加した.    
  \item[STD-HR, HR-9] テグス溶断線過熱による要員負傷防止を,物理カバーによる遮断から
    電源インヒビットによる過熱防止へ変更.(機体寸法上の制約から)
  \item[UHR-2] 展開機構の意図せぬ展開において,要員へのダメージが全て
    マージナルハザードとなった.(膜展開部落下による足の負傷,及び展開
      部が目に当たることによる負傷は,作業時の一般的な服装,靴,保護眼
      鏡で十分避けられるため.)
  \item[UHR-3] 電波放射による要員負傷のハザードをクリティカルハザード
    からマージナルハザードへ変更.(要員負傷の条件について JMR-002B に基づ
      く安全距離からではなく,CSA-109013 に基づくハザード識別基準に変
      更したことによる)
\item[UHR-4] バッテリが放出まで保管温度環境内にあることの担保を熱真空
  試験によるものからメーカ仕様書確認へ変更.(ICDより環境温度が常にバッテ
    リ許容温度範囲内となった為)

\end{description}

\section{審査結果}
各システム安全文書について以下の指摘A/I処置を持って,対処・検証結果は適切であ
るとの判定を受けた.

\begin{itemize}
  \item OrigamiSat-1は射場作業(引渡し時の確認作業を除く)が無いことが
    確認された為,射場作業に関するハザード識別を削除.
  \item リチウムポリマー電池については,電解液のリークが起きうるため,
    これによる受傷をハザードとして UHR-4 に追加し,タイトルも「バッテ
      リの破裂・発火・電解液リーク」に変更.
    \item バッテリ内部短絡(UHR-4)の制御方法として 「UL1642 認証品を用
      いる」旨が認められたため,これに変更する.
\end{itemize}

