\section{Phase 0/1}

Phase 0/I では,主に,SDP の準備(システム安全計画書の作成,各ハザード
  の識別)の妥当性について検証された.本衛星では,スタンダードハザード
(OP-S1-0041 OrigaiSat-1 STD HR) 
以外のユニークハザードとして,以下を識別した.(括弧内はハザード文書名)
次節以降,各ハザード識別概要をまとめる.詳細については各文書を参照され
たい.

\begin{itemize}
 \item 打上振動による構造破壊 (OP-S1-0042 OrigamiSat-1 UHR-1)
 \item 放出前の誤展開 (OP-S1-0043 OrigamiSat-1 UHR-2)
 \item 電波放射 (OP-S1-0050 OrigamiSat-1 UHR-3)
  \item バッテリ短絡 (OP-S1-0053 OrigamiSat-1 UHR-4)
\end{itemize}

\subsection{打上振動による構造破壊(UHR-1)}
本 UHR では,ロケット搭載時に振動や衝撃により衛星構造が破損した場合に
ついて考慮した.上記の構造破壊が発生した場合,破片がロケット構造へダメー
ジを与えることが懸念される.また,放出機構 (E-SSOD) から衛星が正しく放
出されず,ロケット構造への衝突や運用への影響を与えること(カタストロ
  フィックハザード)が起こり得ると判断し
た.

ハザードの制御対応としては,振動・衝撃に対する構造解析・設計,打上時の内圧変化に対す
るベントホール解析,適切な材料の選定,累積疲労・組立の管理によって行う
こととした.

\subsection{展開機構の意図しない展開 (UHR-2)}
本衛星は,伸展カメラ機構,膜展開機構,展開アンテナの3つの展開機構を有している.
これらが正しいタイミング以外で展開した場合に起こり得るハザードを識別し
た(いずれもカタストロフィックハザード).伸展カメラ機構については,地上作業時に保持解放機構が誤動作により解放された場合,膜
展開部 (約 1kg) が落下し要員の足を負傷させる可能性があること,また,放出機構から
の適切な放出を阻害し,ロケットへダメージを与える可能性があることを識別した.膜展開
機構については,地上作業時に保持展開機構が誤動作により解放された場合に
展開ブームの先端が高速で要員に衝突し負傷させる可能性と,放出機構内で誤
動作により展開した場合に正常な放出が阻害され,ロケットにダメージを与え
る可能性について識別した.展開アンテナについても同様に,作業中の要員へ
の傷害,正常な放出の阻害について識別した.

主たる原因としては,各展開構造および保持解放機構,保持テグスの破壊,保持解放機構の
誤動作を識別し,強度設計,保持テグスの適切な選定・管理,および電源供給
の三重遮断 (3 インヒビット)によりハザード制御を行うこととした.特にテ
グスについては,切断・緩み・伸びと起こり得る不具合が多いため,シャープ
エッジの除去,十分に延ばしたテグスの使用,ボビンを用いた緩み除去が容易
なテグス基部等の対策を行った.

\section{意図しない電波放射 (UHR-3)}\label{5_1_UHR3}
本衛星は,UHFと5.8GHzの二種の送信アンテナを搭載している.これらが衛星
放出前に電波を放射すると,ロケットに誤動作・故障等の影響を与える可能性
がある (カタストロフィックハザードと識別).また,ポッド収納作業時において,作業要員が電波放射時の安全距離
(JMR-002B に基づき,UHF: 30cm, 5.8GHz: 35cm と計算)以内に入る為,意
図せぬ電波放射時に負傷する可能性がある(クリティカルハザードと識別).

発生は意図せぬ通電が原因となるため,ハザード制御方法として,ポッド格納
時は放出検知スイッチ(メカニカルスイッチ)による電源3インヒビット,ポッ
ド収納作業時においては,フライトピンを用いた電源3インヒビットを用意し
た.フライトピンは,放出検知スイッチを物理固定するピン2つと,EPSに搭載
されているジャンパピンによるもの1つを用いた.

\subsection{バッテリの破裂・発火(UHR-4)}
衛星引渡しから,衛星放出までの期間にバッテリが破裂・発火し,ロケットや
要員へダメージを与える可能性についてカタストロフィックハザードとして識
別した.

ハザードの要因としては,バッテリ過充電,バッテリ内部短絡,バッテリ外部
短絡,バッテリセルの密閉不良を識別し,それぞれ以下の対策を実施すること
としした.

バッテリ過充電については,本衛星は引き渡し後の充電は行わない為,太陽電
池からの充電遮断が対策となり,第\ref{5_1_UHR3}節と同様に放出検知スイッ
チ及びフライトピンによる3インヒビットとした.また,内部短絡については
UN適合バッテリの使用,及び環境試験前後の充放電特性確認により対策するこ
ととした.外部短絡については,使用バッテリに内蔵されているPTCヒューズ
と回路の絶縁処理による対策,又は,バッテリセルからインヒビッ
  トまでの回路を二重絶縁とする対策を取ることとした.(後にバッテリユニット内の回
  路について絶縁距離が確保できない事が判明し,前者を採用)購入バッテリユニット内部の絶
縁処理については,メーカーの証明書により確認することとした.バッテリセ
ルの密閉不良については,フライト実績のあるバッテリの使用と,バッテリセ
ルへ機械的ストレスを与えない構造および引き渡し後放出までバッテリ保管温度範囲を逸脱しない
設計とすることを対策とした.

\subsection{審査結果}
上記のハザード識別および対処・検証方法は適切であるとの判定を受けた.本
審査において,ハザードの被害対象に「主衛星」を含まないこととなった(本ミッションは
  ピギーバックミッションではないという扱いとなったため).また,UHR-1
について搭載ポッドへの引っかかりについて,構造破壊以外の要因(質量・重
  心位置・搭載磁石の影響)についても検討を要すると指摘を受けた.これら
は,ロケットインターフェース記載の寸法・重量特性を順守すれば問題無いこ
とをロケット側と確認した.尚,電波放射については,CSA-109013「JMR-002B
5.7(3)電波放射系の人体に対する防護」の基準に沿って,ハザードレベルを緩
和できる可能性があること(Phase II で対応),および,本審査会以後に制定予定の CS-108024
「民生用バッテリの安全設計ガイドライン」に基づき,UL1642認定があれば,
UHR-4の短絡保護機能の検証結果提出に代えられる予定である旨(Phase III
  で対応)の助言を受け
た.
