\section{運用の経緯と軌道上データ(坂本・中西)}

打ち上げ直後,最初のパス(2019年1月18日午前)は仰角が浅すぎて地上局で待ち構えていなかったが,アマチュア無線家より打ち上げ直後にOrigamiSat-1のコールサインを受信した旨がTwitterに報告された.
東工大では2019年1月18日午後のパスでわずかにCWが聞こえ,溶断停止コマンドを送信したところ,その後の衛星からのレスポンスをアマチュア無線家が受信してくれていた.これにより,OrigamiSat-1の地上局との送信・受信というミニマムサクセスを達成した.また,この通信の達成により,Radio Amateur Satellite Corporation (AMSAT)から,OrigamiSat-1に対しオスカーナンバーFO-98 (Fuji Oscar 98)が与えられた.

CWの受信が弱かったのは,TLEが確定しておらずどのObjectを追尾するべきか不明だったことと,OrigamiSat-1メンバーが地上局の操作の習熟度が低かったことがある.アマチュア無線家はTLEが未確定の中でも明晰な電波を受信しており,大幅な技術力の差があった.
以下では運用の経緯と,得られた軌道上データを掲載する.

\subsection{運用の経緯}
打ち上げ以降の出来事を以下に時系列にまとめる.





\subsection{軌道上データ}

